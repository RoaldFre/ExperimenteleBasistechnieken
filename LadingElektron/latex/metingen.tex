\section{Metingen}
\begin{table}
\caption{Waargenomen straal van de elektronenbaan}
\label{tabel-elektron}
\begin{center}
\begin{tabular}{c|c||r@{.}l|r@{.}l||r@{.}l|r@{.}l||r@{.}l|r@{.}l}
\multicolumn{1}{c|}{$V$ (V)}&
\multicolumn{1}{c||}{$\Delta V$ (V)}&
\multicolumn{2}{c|}{$I$ (A)}&
\multicolumn{2}{c||}{$\Delta I$ (A)}&
\multicolumn{2}{c|}{$r$ (mm)}&
\multicolumn{2}{c||}{$\Delta r$ (mm)}&
\multicolumn{2}{c|}{$\mathcal{Q}$ (TC/kg)}&
\multicolumn{2}{c}{$\Delta \mathcal{Q}$ (TC/kg)}\\\hline
175&	3&	3&00&	0&05&	17&8&	0&8&	0&20&	0&02\\
175&	3&	2&50&	0&04&	22&5&	0&8&	0&18&	0&01\\
175&	3&	2&45&	0&04&	22&8&	0&8&	0&19&	0&01\\
175&	3&	2&00&	0&03&	27&5&	0&8&	0&19&	0&01\\
175&	3&	1&85&	0&03&	30&8&	0&8&	0&18&	0&01\\
175&	3&	1&50&	0&03&	35&8&	0&8&	0&20&	0&01\\
175&	3&	1&00&	0&03&	56&9&	0&8&	0&17&	0&01\\\hline
250&	4&	1&10&	0&03&	61&3&	0&8&	0&18&	0&01\\
250&	4&	1&30&	0&03&	52&5&	0&8&	0&18&	0&01\\
250&	4&	1&50&	0&03&	44&3&	0&8&	0&19&	0&01\\
250&	4&	1&70&	0&03&	37&8&	0&8&	0&20&	0&01\\
250&	4&	1&90&	0&03&	35&0&	0&8&	0&19&	0&01\\
250&	4&	2&10&	0&03&	32&3&	0&8&	0&18&	0&01\\
250&	4&	2&30&	0&03&	28&8&	0&8&	0&19&	0&01\\
250&	4&	2&50&	0&04&	24&8&	0&8&	0&22&	0&02\\
250&	4&	2&75&	0&04&	18&3&	0&8&	0&33&	0&03\\
250&	4&	3&00&	0&05&	19&3&	0&8&	0&25&	0&02\\\hline
265&	4&	1&75&	0&03&	38&8&	0&8&	0&19&	0&01\\
225&	3&	1&75&	0&03&	36&0&	0&8&	0&19&	0&01\\
200&	3&	1&75&	0&03&	33&3&	0&8&	0&19&	0&01\\
175&	3&	1&75&	0&03&	31&3&	0&8&	0&19&	0&01\\
150&	3&	1&75&	0&03&	30&0&	0&8&	0&18&	0&01\\
125&	3&	1&75&	0&03&	27&0&	0&8&	0&18&	0&01\\
100&	3&	1&75&	0&03&	24&0&	0&8&	0&19&	0&02\\\hline
245&	4&	1&10&	0&03&	60&5&	0&8&	0&18&	0&01\\
225&	3&	1&10&	0&03&	58&3&	0&8&	0&18&	0&01\\
200&	3&	1&10&	0&03&	55&3&	0&8&	0&18&	0&01\\
175&	3&	1&10&	0&03&	51&0&	0&8&	0&18&	0&01\\
150&	3&	1&10&	0&03&	47&3&	0&8&	0&18&	0&01\\
125&	3&	1&10&	0&03&	42&0&	0&8&	0&19&	0&01\\
100&	3&	1&10&	0&03&	37&5&	0&8&	0&19&	0&01\\
\end{tabular}
\end{center}
\end{table}


Er werden vier reeksen van metingen verricht. De eerste twee metingen 
werden uitgevoerd bij constante spanning. De stroom werd gevari\"eerd 
waarbij de straal telkens gemeten werd. Bij de laatste twee reeksen werd 
daarentegen de stroom constant gehouden. De resultaten van de metingen zijn 
ondergebracht in tabel \ref{tabel-elektron}.

Volgens het afgeleide model (\ref{em-equation}) is de straal recht 
evenredig met de vierkantswortel van het aangelegde spanningsverschil en 
omgekeerd evenredig met de stroom. Via een lineaire regressie met 
respectievelijk de vierkantswortel van de aangelegde spanning en de inverse 
van de stroom is dit verband te onderzoeken. Theoretisch gezien zou de 
intercept met de $x$-as gelijk moeten zijn aan nul.  De bekomen waarden 
voor de metingen bij constante spanning zijn
$$
a_1 = (0.02 \pm 0.02)\,(\textrm{Am})^{-1}
\qquad \textrm{en} \qquad
a_2 = (0.08 \pm 0.02)\,(\textrm{Am})^{-1}
$$
met $a_1$ de offset voor de meting bij een spanning van $175$\,V en $a_2$ de 
offset voor de meting bij $250$\,V. Bij de meting met constante stroom horen de 
waarden
$$
a_3 = (-0.4 \pm 0.4)\,\textrm{V}^{1/2}\textrm{/m}
\qquad \textrm{en} \qquad
a_4 = (1.0 \pm 0.3)\,\textrm{V}^{1/2}\textrm{/m}
$$
waarbij $a_3$ de offset is voor de meting bij een constante stroom van 
$1.75$\,A en $a_4$ deze bij een stroom van $1.1$\,A.




Merk op dat een lineaire regressie enkel rekening houdt met de statistische 
fout. Er komt echter nog een behoorlijke fout bij ten gevolge van een fout 
op de metingen, zoals ook later zal blijken voor de gevonden 
$\mathcal{Q}$-waarde.

De intercepts blijken hier, met inachtneming van de instrumentele fouten, 
klein genoeg dat we mogen aannemen dat ze nul zijn.

Daarom
