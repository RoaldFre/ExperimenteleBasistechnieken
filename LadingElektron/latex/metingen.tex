\section{Metingen}

Er werden vier reeksen van metingen verricht. De eerste twee metingen 
werden uitgevoerd bij constante spanning. De stroom werd gevari\"eerd 
waarbij de straal telkens gemeten werd. Bij de laatste twee reeksen werd 
daarentegen de stroom constant gehouden. De resultaten van de metingen zijn 
ondergebracht in tabel \ref{tabel-elektron}.

Volgens het afgeleide model (\ref{em-equation}) is de straal recht 
evenredig met de vierkantswortel van het aangelegde spanningsverschil en 
omgekeerd evenredig met de stroom. Via een lineaire regressie met 
respectievelijk de vierkantswortel van de aangelegde spanning en de inverse 
van de stroom is dit verband te onderzoeken. Theoretisch gezien zou de 
intercept met de $y$-as gelijk moeten zijn aan nul.  De bekomen waarden 
voor de metingen bij constante spanning zijn
$$
a_1 = (0.02 \pm 0.02)\,(\textrm{Am})^{-1}
\qquad \textrm{en} \qquad
a_2 = (0.08 \pm 0.02)\,(\textrm{Am})^{-1}
$$
met $a_1$ de offset voor de meting bij een spanning van $175$\,V en $a_2$ de 
offset voor de meting bij $250$\,V. Bij de meting met constante stroom horen de 
waarden
$$
a_3 = (-0.4 \pm 0.4)\,\textrm{V}^{1/2}\textrm{/m}
\qquad \textrm{en} \qquad
a_4 = (1.0 \pm 0.3)\,\textrm{V}^{1/2}\textrm{/m}
$$
waarbij $a_3$ de offset is voor de meting bij een constante stroom van 
$1.75$\,A en $a_4$ deze bij een stroom van $1.10$\,A.

Merk op dat uit de fouten van (twee van deze) waarden volgt dat het heel 
onwaarschijnlijk is dat de intercept nul zou zijn. Een lineaire regressie 
houdt echter enkel rekening met de statistische fout. Er komt nog een 
behoorlijke fout bij ten gevolge van een fout op de meetgegevens, zoals ook 
later zal blijken voor de gevonden $\mathcal{Q}$-waarde. De intercepts blijken 
hier, met inachtneming van de instrumentele fouten, klein genoeg opdat mag 
worden aangenomen dat ze nul zijn.

Hierom kan men het model fitten via een lineaire regressie met de intercept op 
nul. Zo wordt enkel de helling van de beste fit verkregen. Hieruit kan de 
experimentele $\mathcal{Q}$-waarde berekend worden. De resultaten van deze fits 
staan afgebeeld in figuren \ref{constanteV} voor de metingen met constante 
spanning en \ref{constanteI} voor de waarnemingen met constante stroom.

\figuurOctaveTwee[h!]{constanteV}{Metingen met constante
spanning}{reeks1}{175\,V}{reeks2}{250\,V}
\figuurOctaveTwee[h!]{constanteI}{Metingen met constante stroom}{reeks3}{
1.75\,A}{reeks4}{1.10\,A}

De soortelijke lading wordt uitgedrukt in teracoulomb per kilogram. De waardes 
voor de soortelijke lading $\mathcal{Q}$ op deze manier bekomen zijn
$$
\mathcal{Q}_1 = (0.185 \pm 0.005)\,\textrm{TC/kg}
\qquad \textrm{en} \qquad
\mathcal{Q}_2 = (0.191 \pm 0.008)\,\textrm{TC/kg}
$$
waarbij $\mathcal{Q}_1$ de waarde is voor de meting bij een constante spanning 
van $175$\,V en $\mathcal{Q}_2$ de waarde voor metingen bij een constante 
spanning van $250$\,V. De waarden voor de soortelijke lading bij constante 
stroom zijn
$$
\mathcal{Q}_3 = (0.188 \pm 0.007)\,\textrm{TC/kg}
\qquad \textrm{en} \qquad
\mathcal{Q}_4 = (0.18 \pm 0.01)\,\textrm{TC/kg}
$$
hierbij is $\mathcal{Q}_3$ waargenomen bij een stroom van $1.75$\,A en 
$\mathcal{Q}_4$ bij een stroom van $1.10$\,A.

Merk op dat deze fout ook hier enkel de statistische fout is en geen rekening 
houdt met de fouten op de meetgegevens. Deze fout moet correcter geschat 
worden.  In tabel \ref{tabel-elektron} is voor elk meetpunt een waarde voor 
$\mathcal{Q}$ en de afwijking erop berekend door middel van de fout op de 
meetwaarden. Hieruit kan men besluiten dat de fout op het uiteindelijke 
resultaat minstens 0.01\,TC/kg is.  Het gemiddelde van de waarden is onze 
uiteindelijke $\mathcal{Q}$ waarde.
\begin{center}
{\large
\fbox{$\mathcal{Q} = (0.19 \pm 0.01)\,\textrm{TC/kg}$}
}
\end{center}

Een vergelijking met de literatuurwaarde ($\mathcal{Q}_{\textrm{lit}} = 
0.18\,\textrm{TC/kg}$) leert ons dat de beschreven methode een vrij correcte 
waarde geeft voor de soortelijke lading van het elektron. Sterk opvallend is 
dat de bekomen waarde in bijna elk geval een overschatting is (cf. tabel 
\ref{tabel-elektron}).
Voor deze systematische afwijking zijn enkele verklaringen mogelijk. Zo was de 
Helmholtz spoel bijvoorbeeld niet stabiel gemonteerd. Door lichtjes druk uit te 
oefenen op het apparaat verschoof de elektronenstraal zienderogen. Ook waren de 
omstandigheden om waarnemingen in te verrichten niet ideaal. De 
elektronenstraal heeft bijvoorbeeld een zeer lage intensiteit. Maar om de 
kathetometer correct te kunnen richten, moet men een richtkruis kunnen 
onderscheiden. Het is om die redenen goed mogelijk dat er een relatief grote 
systematische uitwijking is op de uitgelezen elektronenstraal.
