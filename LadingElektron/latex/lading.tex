\documentclass[11pt,a4paper]{article}
\usepackage{amsfonts}
\usepackage{amsmath}
\usepackage{amssymb}
\usepackage[dutch]{babel}
\usepackage{subfigure}
\usepackage[pdftex]{graphicx}
\usepackage{epstopdf}
\usepackage{wasysym}

\usepackage{a4wide}

\author{Roald Frederickx\\Kasper Meerts}
\title{De soortelijke lading van het elektron}
\date{26 april 2010}

\renewcommand{\baselinestretch}{1.17}

\newcommand{\partieel}[2]{\frac{\partial #1}{\partial #2}}
\newcommand{\partpart}[2]{\frac{\partial^2 #1}{\partial #2^2}}
\newcommand{\grad}{^\circ}

\newcommand{\freso}{f_\mathrm{reso}}

\newcommand{\figuur}[4][htb]{
    \begin{figure}[#1]
        \begin{center}
	    \includegraphics[#2]{#3}\\
	    %\parbox{#2}{\caption{#4\label{#3}}}
	    %voor: \includegraphics[width=#2]{#3}\\
	    \caption{#4\label{#3}}
        \end{center}
    \end{figure}}
%\figuur[htb]{width=breedte+eenheid}{naam=label}{caption}


% \figuurOctave[htb]{naam=label}{caption}
\newcommand{\figuurOctave}[3][htb]{
    \begin{figure}[#1]
        \begin{center}
		\nonstopmode
		\input{afbeeldingen/#2.tex}
		\errorstopmode
		\caption{#3\label{#2}}
        \end{center}
    \end{figure}}

% \figuurOctaveTwee[htb]{globaal label}{globale caption}
% 	{naam1=label1}{caption1}{naam2=label2}{caption2}
\newcommand{\figuurOctaveTwee}[7][htb]{
\begin{figure}[#1]
\begin{center}
\hspace{-4cm} %hack om iets breder dan \textwidth te kunnen centreren
\subfigure[#5]{ %sub-caption
	\scalebox{0.9}{ %ietsje verkleinen zodat getallen niet huge
		\nonstopmode
		\input{afbeeldingen/#4.tex}
		\errorstopmode
		\label{#4}
		\rule[-0.8cm]{0cm}{0cm} %voorkom overlap met sub-caption
	}
}
%
\rule{0.6cm}{0cm} %spacer zodat axis labels niet overlappen
%
\subfigure[#7]{
	\scalebox{0.9}{ %ietsje verkleinen zodat getallen niet huge
		\nonstopmode
		\input{afbeeldingen/#6.tex}
		\errorstopmode
		\label{#6}
		\rule[-0.8cm]{0cm}{0cm} %voorkom overlap met sub-caption
	}
}
\hspace{-4cm} % centreer hack
\caption{#3\label{#2}}
\end{center}
\end{figure}
}

\begin{document}
\graphicspath{{"./afbeeldingen/"}}
\maketitle

\section{Doelstellingen}

In deze proef wordt de soortelijke lading van het elektron berekend. Dit is de 
verhouding van de lading van het elektron tot diens massa. Dit is mogelijk door 
de straal van de cirkelbaan te observeren die een elektron aflegt in een 
magnetisch veld. 

Hierbij wordt een uniform magnetisch veld opgewekt door een Helmholtz-spoel.  
Door het vari\"eren van de stroom die door de spoel of de spanning die de 
elektronen versnelt proberen we zo de meting uit te voeren.

\section{Voorbereiding}

\subsection{Helmholtz spoel}

Een Helmholtz spoel is een apparaat dat gebruikt wordt om zeer uniforme 
magnetische velden aan te leggen. Het is opgebouwd uit twee spoelen met straal 
$R$ en $N$ windingen. Ze bevinden zich op een afstand $R$ van elkaar,
hetzelfde als de straal. Door beide spoelen stroomt dezelfde stroom $I$. De 
opstelling staat afgebeeld in figuur \ref{helmholtz}.

%\figuur{width=0.4\textwidth}{helmholtz}{Schematische weergave van een 
%Helmholtz spoel}

Het veld van de opstelling is de som van de velden van twee aparte spoelen. Het 
veld van \'e\'en enkele spoel op de centrale as kan berekend worden met de wet 
van Biot-Savart.
$$
\vec{B} = \frac{\mu_0 I}{4 \pi} \int \frac{\vec{dl} \times \hat{r}}{r^2}
$$
Vanuit symmetrieoverwegingen kan het $B$-veld enkel volgens de $x$-as zijn. Het 
veld op een afstand $h$ van het centrum wordt dus
$$
B_x = \frac{\mu_0 I}{4 \pi} \frac{2 \pi R}{r} \frac{R}{r}
$$
Hier wordt de uitdrukking voor $r$, de afstand tot een elementje van de spoel, 
ingevuld en de vergelijking wordt vereenvoudigd
$$
B = \frac{\mu_0 I R^2}{2(R^2+x^2)^{3/2}}
$$
Het veld in de Helmholtz spoel kan nu berekend door het veld van twee spoelen 
op te tellen. Omdat we het veld in het midden van het apparaat nodig hebben, 
stellen we $x$ gelijk aan $R/2$
$$
B = 2 \frac{\mu_0 I R^2}{{2(R^2+(R/2)^2})^{3/2}}
$$
Na vereenvoudiging bekomt men
$$
B = \left(\frac{4}{5}\right)^{3/2} 
$$

\subsection{Baan van een elektron in een uniform magnetisch veld}

De beginsnelheid van het elektron kan bepaald worden door de wet van behoud van 
energie.  De elektronen worden vanuit rust versneld door het aangelegde 
spanningsverschil.  De kinetische energie kan dus beschreven worden door
$$
\frac{mv^2}{2} = eV
$$
Oplossen naar de snelheid geeft
$$
v = \sqrt{ \frac{e}{m} 2V }
$$
De kracht op een bewegende lading $q$ in een elektromagnetisch veld wordt 
gegeven door de Lorentz-kracht
$$
\vec{F} = q(\vec{E} + \vec{v} \times \vec{B})
$$
In deze proef is enkel een magnetisch veld aanwezig. De lading van het elektron 
wordt voorgesteld door $e$. Tenslotte worden de elektronen loodrecht op de 
magnetische veldlijnen ge\"injecteerd. De uitdrukking voor de kracht wordt dan
$$
F = e v B
$$
Deze kracht staat altijd loodrecht op zowel de snelheid van het deeltje als het 
magnetisch veld. Omdat de kracht altijd loodrecht op de snelheid staat, blijft 
de grootte van de snelheid constant en de grootte van de kracht dus ook. De 
elektronen voeren een cirkelbeweging uit. In een cirkelbaan is de centrifugale 
kracht even groot als de centripetale kracht, in dit geval de magnetische 
kracht.
$$
\frac{mv^2}{r} = evB
$$
Waarbij $r$ de straal is van de cirkelbaan.
Daar de snelheid van de elektronen altijd onder 10\,\% van de lichtsnelheid 
blijft, mogen relativistische effecten verwaarloosd worden. Deze vergelijking 
wordt opgelost naar de soortelijke lading $e/m$
$$
\frac{e}{m} = \frac{v}{Br}
$$
Hier kan de uitdrukking voor snelheid rechtsreeks ingevuld worden omdat de 
snelheid constant blijft.
$$
\frac{e}{m} = \sqrt{\frac{e}{m} \frac{2V}{B^2r^2}}
$$
Hieruit volgt de uitdrukking voor de soortelijke lading.
$$
\frac{e}{m} = \frac{2V}{B^2r^2}
$$


%\section{Numerieke simulatie}
Als extra voorbereiding werd een numerieke simulatie uitgevoerd. Alle 
bovenstaande plots zijn hier resultaten van.

De simulatie werd opgezet met waarden voor de parameters die overeenkomen 
met de experimentele waarden in de volgende sectie.

Voor de eenvoud werd wel gekozen om de sinusburst te moduleren door een 
herhaalde gaussische, en niet door een blokgolf. Dit laat toe om 
dispersieverschijnselen eenvoudiger te simuleren. Concreet werd de 
dispersie ge\"implementeerd door de standaarddeviatie van de omhullende 
gaussische met een constante snelheid in de tijd te laten verbreden. Een 
impressie van de complexere dispersieverschijnselen die we opmerkten bij 
een blokvormige modulatie (zoals gebruikt in het experiment) is geschetst 
in figuur \ref{dispersie}.

Per reflectie wordt de breedte van de gaussische zo met een constante 
waarde verhoogd. Bovendien wordt de hoogte van een puls per `trip` (dus per 
twee reflecties) verlaagd door te vermenigvuldigen met een 
reflectieco\"effici\"ent. In de simulatie werd hiervoor een waarde van 
$0.7$ gekozen, dit komt ruwweg overeen met de experimenteel geobserveerde 
waarde voor water.

Hoe \'e\'en zulke puls evolueert wanneer ze heen en weer wordt 
gereflecteerd in de resonantiekamer (en telkens wordt opgevangen door de 
sensor) werd weergegeven in figuur \ref{puls-echo}. Merk op dat voor de 
duidelijkheid van de grafiek de frequentie van de snelle sinusgolf hier 
vier maal lager is dan de werkelijk gebruikte frequentie in het experiment.

Een compleet golfpakket met interferentie van haar reflecties wordt dan 
opgebouwd door een burst op te tellen met haar reflecties (met grotere 
breedte en lagere hoogte) die werden verschoven over een kleine 
delayperiode afhankelijk van het verschil tussen de repetitiefrequentie van 
de bursts en de resonantiefrequentie. De opbouw van zo'n pakket wordt 
weergegeven in figuur \ref{comboPakket}. Onderaan staat het `eindpakket' 
dat bekomen is door optellen van de bovenstaande gereflecteerde bursts (in 
het vervolg van de simulatie zal steeds gewerkt worden met een som van 20 
reflecties). De afname in amplituden ten gevolge van de 
reflectieco\"effici\"ent en de verbreding van de puls ten gevolge van 
dispersie zijn duidelijk merkbaar per reflectie.

\figuurOctave{comboPakket}{Samenstellen van een golfpakket door 
interferentie tussen de uitgezonde puls en haar reflecties}

In wat volgt zal met `delay' steeds het tijdsverschil bedoeld tussen het 
moment waarop een \'e\'en maal gereflecteerde puls aankomt bij een nieuw 
uitgezonde puls ten opzichte van het tijdsverschil voor ditzelfde traject 
voor dezelfde pulsen indien de burstrepetitie zo gekozen is dat deze twee 
pulsen perfect constructief interfereren (bovenstaand besproken figuur 
\ref{comboPakket} zou dit duidelijk moeten maken).

Door nu deze delay te laten vari\"eren rond nul wordt gesimuleerd wat er 
gebeurt indien de burstrepetitiefrequentie vari\"eert rond de 
resonantiefrequentie. In figuur \ref{delayinterferenties} wordt getoond hoe 
het uiteindelijke golfpakket zich dan gedraagt. Wat we zien in het midden 
(met de $y$-index 0) is de resonantiefrequentie (de centrale piek op figuur 
\ref{delayamp}), de twee grote pakketten op $y$-indices $-4$ en $4$ zijn de 
twee pieken naast de centrale piek van figuur \ref{delayamp}. De paketten 
ertussen bevinden zich in het gebied met overheersend destructieve 
interferentie.

\figuurOctave{delayinterferenties}{Gedrag van het uiteindelijke golfpakket 
bij variatie van de delay of burstrepetitiefrequentie}

Uit de maxima van de amplituden van de golfpaketten uit figuur 
\ref{delayinterferenties} is dan ook figuur \ref{delayamp} kunnen worden 
gesimuleerd. De groottes van deze maximale `piekamplituden' werden in deze 
laatste figuur geplot in functie van de delaytijd voor 1 `trip' van een 
burst.

Deze laatste figuur \ref{delayamp} is wat we willen bereiken met de 
metingen. Uit de locatie van de centrale piek kan eenvoudig de 
resonantiefrequentie worden afgelezen waaruit dan meteen de geluidssnelheid 
volgt.



\section{Proefopstelling}

\figuur{width=0.85\textwidth}{opstelling}{Proefopstelling}

De proefopstelling wordt weergegeven in figuur \ref{opstelling}. Het centrale 
deel van de opstelling is de resonantiekamer. Dit is een cilindervormige ruimte 
van precies 5\,cm lang. Aan \'e\'en kant van de cilinder wordt een 
geluidssignaal opgewekt dat aan de andere kant kan opgevangen worden. De 
cilindervlakken reflecteren een groot deel van de puls terug naar de andere 
kant. 

Een computer regelt de uitgangsfrequentie van een sinusgenerator. In dit 
experiment word er gewerkt met frequenties van 13 tot 16\,kHz. Dit signaal 
dient als trigger voor de functiegenerator. De functiegenerator werkt op een 
frequentie van 2.1\,MHz. Het stuurt een burst uit van sinusgolven, 
overeenkomstig met modulatie door een blokgolf. Op de functiegenerator is de 
duty cycle in te stellen door het aantal uitgestuurde cycli in te geven per 
burst.

Voor de puls-echo methode wordt er een enkele puls uitgestuurd aan 1 kHz. Zo 
wordt een continue uitlezing gegarandeerd maar kan een puls ook nooit met de 
volgende puls interfereren. Op de oscilloscoop worden dan waarnemingen gedaan 
naar de tijd tussen elke reflectie.

De andere methode is de resonantiemethode. Er worden pulsen uitgestuurd met een 
duty cycle van ongeveer 15\,\% aan frequenties rond de resonantiefrequentie 
$f_reso$. Door de frequentie van de sinusgenerator op de computer aan te 
passen, kan me zo waargenomen amplitude veranderen. Op deze manier worden alle 
locale maxima gevonden rond de resonantiefrequentie. Met de beschreven methode 
kan zo de geluidssnelheid berekend worden. 

\section{Metingen}
\begin{table}
\caption{Waargenomen straal van de elektronenbaan}
\label{tabel-elektron}
\begin{center}
\begin{tabular}{c|c|r@{.}l|r@{.}l|r@{.}l|r@{.}l}
\multicolumn{1}{c|}{$V$ (V)}&
\multicolumn{1}{c|}{$\Delta V$ (V)}&
\multicolumn{2}{c|}{$I$ (A)}&
\multicolumn{2}{c|}{$\Delta I$ (A)}&
\multicolumn{2}{c|}{$r$ (mm)}&
\multicolumn{2}{c}{$\Delta r$ (mm)}\\\hline
175&	3&	3&00&	0&05&	17&8&	0&2\\
175&	3&	2&50&	0&04&	22&5&	0&2\\
175&	3&	2&45&	0&04&	22&8&	0&2\\
175&	3&	2&00&	0&03&	27&5&	0&2\\
175&	3&	1&85&	0&03&	30&8&	0&2\\
175&	3&	1&50&	0&03&	35&8&	0&2\\
175&	3&	1&00&	0&03&	56&9&	0&2\\\hline
265&	4&	1&75&	0&03&	38&8&	0&2\\
225&	3&	1&75&	0&03&	36&0&	0&2\\
200&	3&	1&75&	0&03&	33&3&	0&2\\
175&	3&	1&75&	0&03&	31&3&	0&2\\
150&	3&	1&75&	0&03&	30&0&	0&2\\
125&	3&	1&75&	0&03&	27&0&	0&2\\
100&	3&	1&75&	0&03&	24&0&	0&2\\\hline
250&	4&	1&10&	0&03&	61&3&	0&2\\
250&	4&	1&30&	0&03&	52&5&	0&2\\
250&	4&	1&50&	0&03&	44&3&	0&2\\
250&	4&	1&70&	0&03&	37&8&	0&2\\
250&	4&	1&90&	0&03&	35&0&	0&2\\
250&	4&	2&10&	0&03&	32&3&	0&2\\
250&	4&	2&30&	0&03&	28&8&	0&2\\
250&	4&	2&50&	0&04&	24&8&	0&2\\
250&	4&	2&75&	0&04&	18&3&	0&2\\
250&	4&	3&00&	0&05&	19&3&	0&6\\\hline
245&	4&	1&10&	0&03&	60&5&	0&2\\
225&	3&	1&10&	0&03&	58&3&	0&2\\
200&	3&	1&10&	0&03&	55&3&	0&2\\
175&	3&	1&10&	0&03&	51&0&	0&2\\
150&	3&	1&10&	0&03&	47&3&	0&2\\
125&	3&	1&10&	0&03&	42&0&	0&2\\
100&	3&	1&10&	0&03&	37&5&	0&2\\
\end{tabular}
\end{center}
\end{table}


Er werden vier reeksen van metingen verricht. De eerste twee metingen 
werden uitgevoerd bij constante spanning. De stroom werd gevari\"eerd 
waarbij de straal telkens gemeten werd. Bij de laatste twee reeksen werd 
daarentegen de stroom constant gehouden. De resultaten van de metingen zijn 
ondergebracht in tabel \ref{tabel-elektron}.

Volgens het afgeleide model (\ref{em-equation}) is de straal recht 
evenredig met de vierkantswortel van het aangelegde spanningsverschil en 
omgekeerd evenredig met de stroom. Via een lineaire regressie met 
respectievelijk de vierkantswortel van de aangelegde spanning en de inverse 
van de stroom is dit verband te onderzoeken. Theoretisch gezien zou de 
intercept met de $x$-as gelijk moeten zijn aan nul.  De bekomen waarden 
voor de metingen bij constante spanning zijn
$$
a_1 = (0.02 \pm 0.02)\,(\textrm{Am})^{-1}
\qquad \textrm{en} \qquad
a_2 = (0.08 \pm 0.02)\,(\textrm{Am})^{-1}
$$
met $a_1$ de offset voor de meting bij een spanning van $XX$\,V en $a_2$ de 
offset voor de meting bij $YY$\,V. Bij de meting met constante spanning 
horen de waarden
$$
a_3 = (-0.4 \pm 0.4)\,\textrm{V}^{1/2}\textrm{/m}
\qquad \textrm{en} \qquad
a_4 = (1.0 \pm 0.3)\,\textrm{V}^{1/2}\textrm{/m}
$$
waarbij $a_3$ de offset is voor de meting bij een constante stroom van 
$XX$\,A en $a_4$ deze bij een stroom van $YY$\,A.




Merk op dat een lineaire regressie enkel rekening houdt met de statistische 
fout. Er komt echter nog een behoorlijke fout bij ten gevolge van een fout 
op de metingen, zoals ook later zal blijken voor de gevonden 
$\mathcal{Q}$-waarde.

De intercepts blijken hier, met inachtneming van de instrumentele fouten, 
klein genoeg dat we mogen aannemen dat ze nul zijn.

Daarom

\section{Besluit}






\end{document}
