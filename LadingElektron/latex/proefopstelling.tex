\section{Proefopstelling}

\figuur{width=0.7\textwidth}{proefopstelling}{Proefopstelling}

De proefopstelling is weergegeven in figuur \ref{proefopstelling}. Links staat 
de glazen buis met een ijle waterstofatmosfeer. Deze staan gemonteerd in een 
Helmholtzspoel met 130 windingen en een straal van 150\,mm. De 
afregelapparatuur staat rechts op de figuur. Het gloeidraad van het 
elektronenkanon wordt opgewarmd via aansluitingen 6 en 7 met een wisselspanning 
van 6.3\,V. De elektronen die loskomen uit de gloeidraad worden versneld door 
een aangelegde spanning via aansluiten 3 en grondaansluiting 4. Deze spanning 
kan worden geregeld tussen 0 en 300\,V. Eventueel kan de elektronenbundel nog 
worden scherpgesteld door een licht negatieve spanning aan te leggen op 
aansluiting 5. De Helmholtz spoel werd van stroom voorzien door een stroombron 
die kan afgesteld worden tussen 0 en 3.0\,A.


