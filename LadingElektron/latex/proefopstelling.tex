\section{Proefopstelling}

\figuur{width=0.7\textwidth}{proefopstelling}{Proefopstelling}

De proefopstelling is weergegeven in figuur \ref{proefopstelling}. Links staat 
een glazen buis met een ijle waterstofatmosfeer. Deze staat gemonteerd in een 
Helmholtzspoel met 130 windingen en een straal van 150\,mm. De 
afregelapparatuur staat rechts op de figuur. De gloeidraad van het 
elektronenkanon is aangesloten via 6 en 7 met een wisselspanning van 6.3\,V. De 
elektronen die loskomen uit de gloeidraad worden versneld door een aangelegde 
spanning via aansluiting 3 en massa aansluiting 4. Deze spanning kan worden 
geregeld tussen 0 en 300\,V. Eventueel kan de elektronenbundel nog worden 
scherpgesteld door een licht negatieve spanning aan te leggen op aansluiting 5.  
De Helmholtz spoel werd van stroom voorzien door een stroombron die kan 
afgesteld worden tussen 0 en 3.0\,A. Deze is aangesloten via 1 en 2. De 
aangelegde spanning wordt gemeten met voltmeter V. De stroom die door de 
Helmholtzspoelen vloeit kan men aflezen op stroommeter A.

\figuur{width=0.4\textwidth}{elektronenkanon}{Detail elektronenkanon}

Het elektronenkanon is in detail weergegeven in figuur \ref{elektronenkanon}.  
De gloeidraad (g) warmt op vanwege de stroom die erdoor vloeit via 
aansluitingen 6 en 7. Hierdoor warmt de kathode (k) op en komen de elektronen 
los uit het metaal. De anode (a) wordt op een positief potentiaal gebracht ten 
opzichte van de kathode via aansluitingen 3 en 4.  De losgekomen elektronen 
worden ernaar toe versneld.  De Wehnelt cilinder (w) wordt gebiasd gehouden op 
een licht negatieve spanning om zo de elektronenbundel te kunnen scherpstellen.

De elektronenbundel wordt op deze manier in de gaskamer ge\"injecteerd. Door 
het magnetisch veld buigt de bundel af. De elektronen bewegen in een cirkelbaan 
zoals beschreven. 
