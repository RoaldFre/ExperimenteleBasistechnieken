\section{Voorbereiding}

\subsection{Helmholtz spoel}

Een Helmholtz spoel is een apparaat dat gebruikt wordt om zeer uniforme 
magnetische velden aan te leggen. Het is opgebouwd uit twee spoelen met straal 
$R$ en $N$ windingen. Ze bevinden zich op een afstand $R$ van elkaar,
hetzelfde als de straal. Door beide spoelen stroomt dezelfde stroom $I$. De 
opstelling staat afgebeeld in figuur \ref{helmholtz}.

%\figuur{width=0.4\textwidth}{helmholtz}{Schematische weergave van een 
%Helmholtz spoel}

Het veld van de opstelling is de som van de velden van twee aparte spoelen. Het 
veld van \'e\'en enkele spoel op de centrale as kan berekend worden met de wet 
van Biot-Savart.
$$
\vec{B} = \frac{\mu_0 I}{4 \pi} \int \frac{\vec{dl} \times \hat{r}}{r^2}
$$
Vanuit symmetrieoverwegingen kan het $B$-veld enkel volgens de $x$-as zijn. Het 
veld op een afstand $x$ van het centrum wordt dus
$$
B = \frac{\mu_0 NI}{4 \pi} \frac{2 \pi R}{r} \frac{R}{r}
$$
Hier wordt de uitdrukking voor $r$, de afstand tot een elementje van de spoel, 
ingevuld en de vergelijking wordt vereenvoudigd
$$
B = \frac{\mu_0 NI R^2}{2(R^2+x^2)^{3/2}}
$$
Het magnetische veld op grote afstand krijgt de vorm
$$
B \approx \frac{\mu_0 NI R^2}{2x^3} = \frac{\mu_0}{4 \pi}\frac{2m}{x^3}
$$
Hier is $m$ het magnetisch dipoolmoment van de spoel, het product van de stroom 
en de oppervlakte van de spoel.

Het veld in de Helmholtz spoel kan berekend door het veld van twee spoelen op 
te tellen.
$$
B = \frac{\mu_0 NI R^2}{2}\left( \frac{1}{(R^2+x^2)^{3/2}} + 
\frac{1}{(R^2+(R-x)^2)^{3/2}} \right)
$$
Een grafiek van het magnetische veld is weergegeven in figuur \ref{Bveld}.

Voor deze proef wordt het veld in het midden gezocht, met $x$ gelijk aan $R/2$.
Invullen en vereenvoudigen geeft
\begin{equation}
\label{Helmholtz-eq}
B = \left(\frac{4}{5}\right)^{3/2} \frac{\mu_0 NI}{R}
\end{equation}

\figuurOctave{Bveld}{De grootte van het magnetisch veld in een Helmholtz spoel.  
De groene stippellijn geeft de grootte van het apparaat aan.}

Dit geeft ons het magnetische veld op de centrale as. Voor de andere plaatsen 
is geen volledig analytische afleiding mogelijk. Deze oplossing wordt hier niet 
beschreven \cite{magnVeldHelmholtz}.
Tekening van de veldlijnen wordt afgebeeld in figuur \ref{B-helmholtz}

\figuur{width=0.6\textwidth}{B-helmholtz}{Het magnetisch veld in een helmholtz 
spoel.}

\subsection{Baan van een elektron in een uniform magnetisch veld}

De beginsnelheid van het elektron kan bepaald worden door de wet van behoud van 
energie. De elektronen worden vanuit rust versneld door het aangelegde 
spanningsverschil $V$. De kinetische energie kan dus beschreven worden door
$$
\frac{mv^2}{2} = eV
$$
Oplossen naar de snelheid geeft
$$
v = \sqrt{ \frac{e}{m} 2V }
$$
De kracht op een bewegende lading $q$ in een elektromagnetisch veld wordt 
gegeven door de Lorentz-kracht
$$
\vec{F} = q(\vec{E} + \vec{v} \times \vec{B})
$$
In deze proef is enkel een magnetisch veld aanwezig. De lading van het elektron 
wordt voorgesteld door $e$. Tenslotte worden de elektronen loodrecht op de 
magnetische veldlijnen ge\"injecteerd. De uitdrukking voor de kracht wordt dan
$$
F = e v B
$$
Deze kracht staat altijd loodrecht op zowel de snelheid van het deeltje als het 
magnetisch veld. Omdat de kracht altijd loodrecht op de snelheid staat, blijft 
de grootte van de snelheid constant en de grootte van de kracht dus ook. De 
elektronen voeren een cirkelbeweging uit. In een cirkelbaan is de centrifugale 
kracht even groot als de centripetale kracht, in dit geval de magnetische 
kracht.
$$
\frac{mv^2}{r} = evB
$$
Waarbij $r$ de straal is van de cirkelbaan.
Daar, voor de experimenteel gebruikte waarden van $B$ en $v$, de snelheid 
van de elektronen altijd onder 10\,\% van de lichtsnelheid blijft, mogen 
relativistische effecten verwaarloosd worden. Deze vergelijking wordt dan
opgelost naar de soortelijke lading $e/m$
$$
\frac{e}{m} = \frac{v}{Br}
$$
Hier kan de uitdrukking voor snelheid rechtsreeks ingevuld worden omdat 
deze constant blijft.
$$
\frac{e}{m} = \sqrt{\frac{e}{m} \frac{2V}{B^2r^2}}
$$
Hieruit volgt de uitdrukking voor de soortelijke lading (aangeduid met 
symbool $\mathcal{Q}$).
$$
\mathcal{Q} = \frac{e}{m} = \frac{2V}{B^2r^2}
$$
Hierin wordt de uitdrukking voor $B$ (\ref{Helmholtz-eq}) ingevuld
\begin{equation}
\label{em-equation}
\mathcal{Q} = \frac{2V}{(4/3)^3} \left( \frac{R}{\mu_0NIr} \right)^2
\end{equation}
