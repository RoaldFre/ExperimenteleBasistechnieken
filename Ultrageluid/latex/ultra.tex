\documentclass[11pt,a4paper]{article}
\usepackage{amsfonts}
\usepackage{amsmath}
\usepackage{amssymb}
\usepackage[dutch]{babel}
\usepackage{subfigure}
\usepackage[pdftex]{graphicx}
\usepackage{epstopdf}
\usepackage{wasysym}

\usepackage{a4wide}

\author{Roald Frederickx\\Kasper Meerts}
\title{Meten van de lichtsnelheid}
\date{19 april 2010}

\renewcommand{\baselinestretch}{1.14}

\newcommand{\figuur}[4][htb]{
    \begin{figure}[#1]
        \begin{center}
	    \includegraphics[#2]{#3}\\
	    %\parbox{#2}{\caption{#4\label{#3}}}
	    %voor: \includegraphics[width=#2]{#3}\\
	    \caption{#4\label{#3}}
        \end{center}
    \end{figure}}
%\figuur[htb]{width=breedte+eenheid}{naam=label}{caption}


% \figuurOctave[htb]{naam=label}{caption}
\newcommand{\figuurOctave}[3][htb]{
    \begin{figure}[#1]
        \begin{center}
		\nonstopmode
		\input{afbeeldingen/#2.tex}
		\errorstopmode
		\caption{#3\label{#2}}
        \end{center}
    \end{figure}}

% \figuurOctaveTwee[htb]{globaal label}{globale caption}
% 	{naam1=label1}{caption1}{naam2=label2}{caption2}
\newcommand{\figuurOctaveTwee}[7][htb]{
\begin{figure}[#1]
\begin{center}
\hspace{-4cm} %hack om iets breder dan \textwidth te kunnen centreren
\subfigure[#5]{ %sub-caption
	\scalebox{0.9}{ %ietsje verkleinen zodat getallen niet huge
		\nonstopmode
		\input{afbeeldingen/#4.tex}
		\errorstopmode
		\label{#4}
		\rule[-0.8cm]{0cm}{0cm} %voorkom overlap met sub-caption
	}
}
%
\rule{0.6cm}{0cm} %spacer zodat axis labels niet overlappen
%
\subfigure[#7]{
	\scalebox{0.9}{ %ietsje verkleinen zodat getallen niet huge
		\nonstopmode
		\input{afbeeldingen/#6.tex}
		\errorstopmode
		\label{#6}
		\rule[-0.8cm]{0cm}{0cm} %voorkom overlap met sub-caption
	}
}
\hspace{-4cm} % centreer hack
\caption{#3\label{#2}}
\end{center}
\end{figure}
}





\begin{document}
\graphicspath{{"afbeeldingen/"}}
\maketitle




\section{De metingen}
[proefopstelling zegt al: 5cm lengte]

Zeggen: modulatie door BLOKgolf, geen gaussische!

\subsection{Verloop van de piekamplituden van het golfpakket nabij de 
resonantiefrequentie}
Eerst werd het gedrag van de piekamplituden van het golfpakket onderzocht 
indien de sinusoidale bursts worden afgevuurd aan een frequentie die in de 
buurt ligt van de resonantiefrequentie. Met `de piekamplitude van het 
golfpakket' wordt de maximale amplitude van een beschouwd golfpakket 
bedoeld, terwijl dit interferentie ondervindt van de reflecties van vorig 
uitgestuurde bursts. In figuur \ref{comboPakket} betekent dit de maximale 
amplitude voor het onderste, samengestelde pakket.

Het doel van deze meting is om te onderzoeken wat de vorm is van de 
omhullende van het patroon (dat een intrinsieke snelle schommeling zal 
kennen door het afwisselen van destructieve en constructieve interferentie) 
en om eventueel dit patroon te parametriseren en zo met behulp van een fit 
een zeer nauwkeurige waarde te bekomen voor de resonantiefrequentie, en dus 
de geluidssnelheid.

Praktisch werd de triggering eerst zo ingesteld dat er aan een frequentie 
van $13.5$\,kHz een burst van van 25 sinusperioden (aan een frequentie van 
$2.1$\,MHz) werd afgevuurd. Op de oscilloscoop was te merken dat deze 
triggeringsfrequentie iets onder de resonantiefrequentie lag. Het 
pulspakket (na interferentie met de gereflecteerde bursts) had immers een 
vorm zoals het onderste pakket in figuur \ref{delayinterferenties}. De 
`linkerstaart' was wel nog wat `zwaarder' dan in die figuur (de frequentie 
lag verder weg van het optimum).

Vervolgens werd de triggeringsfrequentie verhoogd in stappen van 5\,Hz om 
een gedetailleerd beeld te krijgen van het verloop van deze piekamplitude 
bij afwisselend constructieve en destructieve interferentie met de 
weerkaatste golfpaketten.

\figuurOctave[htb]{delayamp-experimenteel}{Visualisatie van de 
piekamplituden bij veranderende burstfrequentie}
Het resultaat hiervan is geplot in figuur \ref{delayamp-experimenteel}.  
Merk op dat dit maal gekozen is om de meetpunten expliciet te verbinden met 
een lijnstuk om het verloop visueel duidelijk te maken.

Uit deze plot volgt inderdaad dat het gekozen frequentiegebied onder de 
resonantiefrequentie lag, de pieken in de gemeten piekamplitude blijven 
immers nog stijgen voor stijgende burstfrequenties.

Omwille van een aantal redenen is gekozen om dit type meting niet verder te 
zetten tot voorbij de resonantiefrequentie. Een eerste reden is puur 
praktisch: deze meting bevat zo'n 130 meetpunten en het verkrijgen van deze 
data via de gebrukte, manuele methode heeft een behoorlijke hoeveelheid 
tijd in beslag genomen. Er was eenvoudigweg onvoldoende tijd om deze 
resolutie in de bekeken burstfrequentie aan te houden tot voorbij de 
resonantiepiek.


Ten tweede zorgt de gebruikte gevoeligheid van de apparatuur ervoor dat 
kleine factoren de bekomen meetwaarden danig kunnen verstoren. Een mooi 
voorbeeld hiervan vinden we terug op de plot in het


Ten laatste zou deze data geen meerwaarde hebben opgeleverd in vergelijking 
met de hieronder beschreven methode die enkel de pieken in deze plot zal 
meten.  Uit de plot blijk immers (net zoals uit de simulatie voor een 
gaussisch gemoduleerde burst) dat het onmogelijk is om de `inwendige' 
oscillaties van te fitten met een eenvoudige sinusfunctie. De toppen lijken 
immers veel sterker te worden versterkt dan de dalen (te merken aan hun 
hogere amplitude en kleinere breedte in de pieken). Een model hiervoor 
opstellen zou ons te ver leiden.



temp door traag


onfitbaar



interferentie nauwkeurig onderzocht. 



\begin{table}[h!t!b!]
\caption{Relatieve piekamplituden bij de resonantiefrequenties}
\label{lucht-tab}
\begin{center}
\begin{tabular}{ r | r@{}l | r@{}l  ||  r | r@{}l | r@{}l }
\multicolumn{1}{c|}{$f$ (Hz) $\pm 4$}&
\multicolumn{2}{c|}{$A$}&
\multicolumn{2}{c||}{$\Delta A$}&
\multicolumn{1}{c|}{$f$ (Hz) $\pm 4$}&
\multicolumn{2}{c|}{$A$}&
\multicolumn{2}{c}{$\Delta A$}\\\hline
12870&	0&.57&	0&.01&	14498&	0&.91&	0&.03\\
12979&	0&.59&	0&.01&	14600&	0&.85&	0&.03\\
13056&	0&.59&	0&.01&	14702&	0&.81&	0&.03\\
13138&	0&.60&	0&.03&	14814&	0&.79&	0&.03\\
13222&	0&.62&	0&.03&	14914&	0&.78&	0&.03\\
13306&	0&.66&	0&.03&	15016&	0&.76&	0&.03\\
13388&	0&.72&	0&.03&	15121&	0&.72&	0&.03\\
13470&	0&.76&	0&.03&	15207&	0&.71&	0&.03\\
13552&	0&.79&	0&.03&	15320&	0&.69&	0&.03\\
13638&	0&.81&	0&.03&	15430&	0&.68&	0&.03\\
13728&	0&.82&	0&.03&	15542&	0&.66&	0&.03\\
13818&	0&.85&	0&.03&	15656&	0&.65&	0&.03\\
13912&	0&.88&	0&.03&	15772&	0&.62&	0&.03\\
14006&	0&.91&	0&.03&	15888&	0&.60&	0&.01\\
14102&	0&.97&	0&.03&	16010&	0&.59&	0&.01\\
14198&	1&.00&	0&.03&	16134&	0&.58&	0&.01\\
14296&	1&.00&	0&.03&	16256&	0&.57&	0&.01\\
14396&	0&.94&	0&.03&	16384&	0&.56&	0&.01\\
\end{tabular}
\end{center}
\end{table}














\end{document}
