\documentclass[11pt,a4paper]{article}
\usepackage{amsfonts}
\usepackage{amsmath}
\usepackage{amssymb}
\usepackage[dutch]{babel}
\usepackage{subfigure}
\usepackage[pdftex]{graphicx}
\usepackage{epstopdf}
\usepackage{wasysym}

\usepackage{a4wide}

\author{Roald Frederickx\\Kasper Meerts}
\title{Meten van de lichtsnelheid}
\date{19 april 2010}

\renewcommand{\baselinestretch}{1.14}

\newcommand{\freso}{f_\mathrm{reso}}

\newcommand{\figuur}[4][htb]{
    \begin{figure}[#1]
        \begin{center}
	    \includegraphics[#2]{#3}\\
	    %\parbox{#2}{\caption{#4\label{#3}}}
	    %voor: \includegraphics[width=#2]{#3}\\
	    \caption{#4\label{#3}}
        \end{center}
    \end{figure}}
%\figuur[htb]{width=breedte+eenheid}{naam=label}{caption}


% \figuurOctave[htb]{naam=label}{caption}
\newcommand{\figuurOctave}[3][htb]{
    \begin{figure}[#1]
        \begin{center}
		\nonstopmode
		\input{afbeeldingen/#2.tex}
		\errorstopmode
		\caption{#3\label{#2}}
        \end{center}
    \end{figure}}

% \figuurOctaveTwee[htb]{globaal label}{globale caption}
% 	{naam1=label1}{caption1}{naam2=label2}{caption2}
\newcommand{\figuurOctaveTwee}[7][htb]{
\begin{figure}[#1]
\begin{center}
\hspace{-4cm} %hack om iets breder dan \textwidth te kunnen centreren
\subfigure[#5]{ %sub-caption
	\scalebox{0.9}{ %ietsje verkleinen zodat getallen niet huge
		\nonstopmode
		\input{afbeeldingen/#4.tex}
		\errorstopmode
		\label{#4}
		\rule[-0.8cm]{0cm}{0cm} %voorkom overlap met sub-caption
	}
}
%
\rule{0.6cm}{0cm} %spacer zodat axis labels niet overlappen
%
\subfigure[#7]{
	\scalebox{0.9}{ %ietsje verkleinen zodat getallen niet huge
		\nonstopmode
		\input{afbeeldingen/#6.tex}
		\errorstopmode
		\label{#6}
		\rule[-0.8cm]{0cm}{0cm} %voorkom overlap met sub-caption
	}
}
\hspace{-4cm} % centreer hack
\caption{#3\label{#2}}
\end{center}
\end{figure}
}

\begin{document}
\graphicspath{{"./afbeeldingen/"}}
\maketitle


\section{Doelstellingen}

In deze proef wordt de soortelijke lading van het elektron berekend. Dit is de 
verhouding van de lading van het elektron tot diens massa. Dit is mogelijk door 
de straal van de cirkelbaan te observeren die een elektron aflegt in een 
magnetisch veld. 

Hierbij wordt een uniform magnetisch veld opgewekt door een Helmholtz-spoel.  
Door het vari\"eren van de stroom die door de spoel of de spanning die de 
elektronen versnelt proberen we zo de meting uit te voeren.

\section{Voorbereiding}
\subsection{Model}

In vervormbare materialen is er voor kleine vervormingen een lineair verband 
tussen de relatieve volumeverandering en de druk nodig voor deze 
volumeverandering. De verhoudingsconstante is de bulkmodulus en is eigen aan 
elk materiaal.
\begin{equation}
\label{bulkmod}
B = -\frac{\Delta P}{\frac{\Delta V}{V}} = - V \frac{\Delta P}{\Delta V}
\end{equation}

\figuur{width=0.7\textwidth}{bulk}{Bulk modulus}

In dit model wordt er een cylinderkolom van een vloeistof beschouwd. De 
uitwijkingen zijn enkel volgens de $x$-co\"ordinaat. De uitwijking van de 
vloeistofdeeltjes oorspronkelijk op co\"ordinaat $x$ en op het tijdstip $t$ 
wordt voorgesteld door $u(x,t)$. Een infinitesimale cylinder van lengte $dx$ en 
oppervlakte $A$ heeft als volume $A dx$. 

De uitwijking van de druk $P$ boven de atmosferische druk $P_0$ is $p$
$$
p = \Delta P = P - P_0
$$
Hiermee kan de uitdrukking voor de bulkmodulus (\ref{bulkmod}) herscheven 
worden in functie van de uitwijking van de druk.
$$
B = -A dx \frac{\Delta P}{\Delta V} = -dx \frac{\Delta P}{ u(x+dx,t)-u(x,t)}
$$
De uitdrukking voor de druk kan hieruit worden ge\"isoleerd
\begin{equation}
\label{bulkmod1d}
p = \Delta P = -B \frac{u(x+dx,t)-u(x)}{dx} = -B \partieel{u}{x}
\end{equation}

De wet van Newton ($F = ma$) voor de cylinder wordt
$$
P(x)A - P(x+dx)A = (p(x) - p(x+dx))A = \rho A dx \partpart{u}{t}
$$
Hierbij is $\rho$ de dichtheid van de vloeistof
$$
\partieel{p}{x} + \rho \partpart{u}{t} = 0
$$
Hierin kan uitdrukking (\ref{bulkmod1d}) gesubstitueerd worden
$$
-B\partpart{u}{x} + \rho \partpart{u}{t} = 0
$$
Na herschrijven wordt dit
\begin{equation}
\label{waveeq}
\partpart{u}{x} - \frac{\rho}{B} \partpart{u}{t} = 0
\end{equation}

Dit is de in de natuurkunde alomtegenwoordige golfvergelijking. De snelheid van 
een golf is eenvoudigweg
\begin{equation}
\label{speedofsound}
v = \sqrt{\frac{B}{\rho}}
\end{equation}

De waardes voor de dichtheid en de bulkmodulus kunnen we opzoeken. Zo kan men 
voor een materiaal de geluidssnelheid berekenen. De resultaten voor enkele 
materialen staan in tabel \ref{speedtable}. 

\begin{table}[htb]
\caption{Berekende geluidsnelheid}
\label{speedtable}
\begin{center}
\begin{tabular}{c||ccc}
Materiaal & $B$ (Pa) & $\rho$ $(\textrm{kg/m}^3)$ & $v$ (m/s) \\\hline
lucht & $1.42 \cdot 10^5$ & 1.2 & 344 \\
water & $2.18 \cdot 10^9$ & 1000 & 1477 \\
glycerol & $4.35 \cdot 10^9$ & 1200 & 1900 \\
staal & $160 \cdot 10^9$ & 7900 & 4500 \\
\end{tabular}
\end{center}
\end{table}

\subsection{Puls-echo methode}
De meest voor de hand liggende manier om de geluidssnelheid te bepalen is meten 
hoe lang het geluid erover doet om een bepaalde afstand te overbruggen. Als de 
tijd van het uitzenden tot het ontvangen van puls $\Delta t$ is en de afstand 
ertussen $L$ is, is de snelheid eenvoudigweg
$$
v = \frac{L}{\Delta t}
$$
In de resonantiekamer kan een enkele puls enkele keren heen en weer blijven 
reflecteren. Telkens reflecteert een vaste fractie van de amplitude, de 
reflectieco"effici"ent. Ook worden de golfpakketjes verstrooid door dispersieve 
effecten. Een typische waarneming werd gesimuleerd en weergegeven in figuur 
\ref{puls-echo}. Een voordeel van het gebruiken van een resonantiekamer is dat
men meerdere waarnemingen in \'e\'en keer kan doen.

\figuurOctave{puls-echo}{Gereflecteerde echo's met een reflectieco"effici\"ent 
van 0.7}

Deze methode heeft slechts een beperkte nauwkeurigheid. Door dispersie 
verstrooien de pulsen waardoor ze moeilijker te localiseren zijn.

\subsection{Staande golf methode}

In een resonantiekamer kan men een verfijndere methode gebruiken. Er worden 
geluidsgolven uitgestuurd van een zeer hoge frequentie, in de orde van enkele 
megahertz. Deze draaggolf wordt gemoduleerd door een blokgolf tot pulsen met 
een frequentie die rond de resonantiefrequentie zit. De uitgezonden puls 
reflecteert heen en weer in de kamer en verliest zijn energie maar dit keer 
worden er ook nieuwe pulsen uitgestuurd. Afhankelijk van wanneer de pulsen 
uitgestuurd worden, interfereren ze constructief of destructief met elkaar. De 
sterkste constructieve interferentie doet zich voor wanneer de frequentie van 
de uitgezonden pulsen \'e\'en van de resonantiefrequenties zijn. Deze 
interferentie doet zich voor wanneer een puls een geheel aantal keer de lengte 
van de resonantiekamer heeft afgelegd en is teruggekeerd voor een nieuwe puls 
wordt uitgestuurd.
\begin{equation}
\label{resfreq}
f_n = \frac{1}{T_n} = \frac{1}{n} \frac{v}{2L}
\end{equation}
In het ideale geval zou elke puls uit slechts \'e\'en cyclus van de draaggolf 
hoeven bestaan. Men zou dan die puls enkel moeten laten overeenkomen met zijn 
reflecties om resonantie waar te nemen. Vanwege praktische redenen is dit 
echter niet mogelijk.

Een puls wordt gedempt door een reflectie. Om de interferentie te kunnen 
waarnemen, zijn we dus aangewezen een grotere hoeveelheid energie uit te zenden 
met elke puls. Ook blijft het geen mooi geheel. Door dispersie wordt de puls 
verstrooid wat de waarnemingen bemoeilijkt. Bredere pulsen hebben hier minder 
last van.

%% ROALD %%
% Figuur van disperserende puls?
%%%%%%%%%%%

Anderzijds mag men ook niet teveel cycli per puls uitzenden. Het wordt dan 
moeilijker om de resonantie-effecten van het uitgezonden signaal te 
onderscheiden. De resonantiepiek wordt ook platter. In de praktijk gebruikt men 
een arbeidscyclus (dutycycle) van 10 \`a 20\,\%.


%%%% KASPER
%%% AFWERKEN
%%% Figuur van amplitudeverloop
%%%%%%




\section{Metingen}
\begin{table}
\caption{Waargenomen straal van de elektronenbaan}
\label{tabel-elektron}
\begin{center}
\begin{tabular}{c|c||r@{.}l|r@{.}l||r@{.}l|r@{.}l||r@{.}l|r@{.}l}
\multicolumn{1}{c|}{$V$ (V)}&
\multicolumn{1}{c||}{$\Delta V$ (V)}&
\multicolumn{2}{c|}{$I$ (A)}&
\multicolumn{2}{c||}{$\Delta I$ (A)}&
\multicolumn{2}{c|}{$r$ (mm)}&
\multicolumn{2}{c||}{$\Delta r$ (mm)}&
\multicolumn{2}{c|}{$\mathcal{Q}$ (TC/kg)}&
\multicolumn{2}{c}{$\Delta \mathcal{Q}$ (TC/kg)}\\\hline
175&	3&	3&00&	0&05&	17&8&	0&8&	0&20&	0&02\\
175&	3&	2&50&	0&04&	22&5&	0&8&	0&18&	0&01\\
175&	3&	2&45&	0&04&	22&8&	0&8&	0&19&	0&01\\
175&	3&	2&00&	0&03&	27&5&	0&8&	0&19&	0&01\\
175&	3&	1&85&	0&03&	30&8&	0&8&	0&18&	0&01\\
175&	3&	1&50&	0&03&	35&8&	0&8&	0&20&	0&01\\
175&	3&	1&00&	0&03&	56&9&	0&8&	0&17&	0&01\\\hline
250&	4&	1&10&	0&03&	61&3&	0&8&	0&18&	0&01\\
250&	4&	1&30&	0&03&	52&5&	0&8&	0&18&	0&01\\
250&	4&	1&50&	0&03&	44&3&	0&8&	0&19&	0&01\\
250&	4&	1&70&	0&03&	37&8&	0&8&	0&20&	0&01\\
250&	4&	1&90&	0&03&	35&0&	0&8&	0&19&	0&01\\
250&	4&	2&10&	0&03&	32&3&	0&8&	0&18&	0&01\\
250&	4&	2&30&	0&03&	28&8&	0&8&	0&19&	0&01\\
250&	4&	2&50&	0&04&	24&8&	0&8&	0&22&	0&02\\
250&	4&	2&75&	0&04&	18&3&	0&8&	0&33&	0&03\\
250&	4&	3&00&	0&05&	19&3&	0&8&	0&25&	0&02\\\hline
265&	4&	1&75&	0&03&	38&8&	0&8&	0&19&	0&01\\
225&	3&	1&75&	0&03&	36&0&	0&8&	0&19&	0&01\\
200&	3&	1&75&	0&03&	33&3&	0&8&	0&19&	0&01\\
175&	3&	1&75&	0&03&	31&3&	0&8&	0&19&	0&01\\
150&	3&	1&75&	0&03&	30&0&	0&8&	0&18&	0&01\\
125&	3&	1&75&	0&03&	27&0&	0&8&	0&18&	0&01\\
100&	3&	1&75&	0&03&	24&0&	0&8&	0&19&	0&02\\\hline
245&	4&	1&10&	0&03&	60&5&	0&8&	0&18&	0&01\\
225&	3&	1&10&	0&03&	58&3&	0&8&	0&18&	0&01\\
200&	3&	1&10&	0&03&	55&3&	0&8&	0&18&	0&01\\
175&	3&	1&10&	0&03&	51&0&	0&8&	0&18&	0&01\\
150&	3&	1&10&	0&03&	47&3&	0&8&	0&18&	0&01\\
125&	3&	1&10&	0&03&	42&0&	0&8&	0&19&	0&01\\
100&	3&	1&10&	0&03&	37&5&	0&8&	0&19&	0&01\\
\end{tabular}
\end{center}
\end{table}


Er werden vier reeksen van metingen verricht. De eerste twee metingen 
werden uitgevoerd bij constante spanning. De stroom werd gevari\"eerd 
waarbij de straal telkens gemeten werd. Bij de laatste twee reeksen werd 
daarentegen de stroom constant gehouden. De resultaten van de metingen zijn 
ondergebracht in tabel \ref{tabel-elektron}.

Volgens het afgeleide model (\ref{em-equation}) is de straal recht 
evenredig met de vierkantswortel van het aangelegde spanningsverschil en 
omgekeerd evenredig met de stroom. Via een lineaire regressie met 
respectievelijk de vierkantswortel van de aangelegde spanning en de inverse 
van de stroom is dit verband te onderzoeken. Theoretisch gezien zou de 
intercept met de $x$-as gelijk moeten zijn aan nul.  De bekomen waarden 
voor de metingen bij constante spanning zijn
$$
a_1 = (0.02 \pm 0.02)\,(\textrm{Am})^{-1}
\qquad \textrm{en} \qquad
a_2 = (0.08 \pm 0.02)\,(\textrm{Am})^{-1}
$$
met $a_1$ de offset voor de meting bij een spanning van $175$\,V en $a_2$ de 
offset voor de meting bij $250$\,V. Bij de meting met constante stroom horen de 
waarden
$$
a_3 = (-0.4 \pm 0.4)\,\textrm{V}^{1/2}\textrm{/m}
\qquad \textrm{en} \qquad
a_4 = (1.0 \pm 0.3)\,\textrm{V}^{1/2}\textrm{/m}
$$
waarbij $a_3$ de offset is voor de meting bij een constante stroom van 
$1.75$\,A en $a_4$ deze bij een stroom van $1.1$\,A.




Merk op dat een lineaire regressie enkel rekening houdt met de statistische 
fout. Er komt echter nog een behoorlijke fout bij ten gevolge van een fout 
op de metingen, zoals ook later zal blijken voor de gevonden 
$\mathcal{Q}$-waarde.

De intercepts blijken hier, met inachtneming van de instrumentele fouten, 
klein genoeg dat we mogen aannemen dat ze nul zijn.

Daarom


\end{document}
