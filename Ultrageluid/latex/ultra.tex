\documentclass[11pt,a4paper]{article}
\usepackage{amsfonts}
\usepackage{amsmath}
\usepackage{amssymb}
\usepackage[dutch]{babel}
\usepackage{subfigure}
\usepackage[pdftex]{graphicx}
\usepackage{epstopdf}
\usepackage{wasysym}

\usepackage{a4wide}

\author{Roald Frederickx\\Kasper Meerts}
\title{Snelheidsmeting ultrasone golven in vloeistoffen}
\date{19 april 2010}

\renewcommand{\baselinestretch}{1.14}

\newcommand{\partieel}[2]{\frac{\partial #1}{\partial #2}}
\newcommand{\partpart}[2]{\frac{\partial^2 #1}{\partial #2^2}}
\newcommand{\grad}{^\circ}

\newcommand{\freso}{f_\mathrm{reso}}

\newcommand{\figuur}[4][htb]{
    \begin{figure}[#1]
        \begin{center}
	    \includegraphics[#2]{#3}\\
	    %\parbox{#2}{\caption{#4\label{#3}}}
	    %voor: \includegraphics[width=#2]{#3}\\
	    \caption{#4\label{#3}}
        \end{center}
    \end{figure}}
%\figuur[htb]{width=breedte+eenheid}{naam=label}{caption}


% \figuurOctave[htb]{naam=label}{caption}
\newcommand{\figuurOctave}[3][htb]{
    \begin{figure}[#1]
        \begin{center}
		\nonstopmode
		\input{afbeeldingen/#2.tex}
		\errorstopmode
		\caption{#3\label{#2}}
        \end{center}
    \end{figure}}

% \figuurOctaveTwee[htb]{globaal label}{globale caption}
% 	{naam1=label1}{caption1}{naam2=label2}{caption2}
\newcommand{\figuurOctaveTwee}[7][htb]{
\begin{figure}[#1]
\begin{center}
\hspace{-4cm} %hack om iets breder dan \textwidth te kunnen centreren
\subfigure[#5]{ %sub-caption
	\scalebox{0.9}{ %ietsje verkleinen zodat getallen niet huge
		\nonstopmode
		\input{afbeeldingen/#4.tex}
		\errorstopmode
		\label{#4}
		\rule[-0.8cm]{0cm}{0cm} %voorkom overlap met sub-caption
	}
}
%
\rule{0.6cm}{0cm} %spacer zodat axis labels niet overlappen
%
\subfigure[#7]{
	\scalebox{0.9}{ %ietsje verkleinen zodat getallen niet huge
		\nonstopmode
		\input{afbeeldingen/#6.tex}
		\errorstopmode
		\label{#6}
		\rule[-0.8cm]{0cm}{0cm} %voorkom overlap met sub-caption
	}
}
\hspace{-4cm} % centreer hack
\caption{#3\label{#2}}
\end{center}
\end{figure}
}

\begin{document}
\graphicspath{{"./afbeeldingen/"}}
\maketitle


\section{Doelstellingen}

In deze proef wordt de soortelijke lading van het elektron berekend. Dit is de 
verhouding van de lading van het elektron tot diens massa. Dit is mogelijk door 
de straal van de cirkelbaan te observeren die een elektron aflegt in een 
magnetisch veld. 

Hierbij wordt een uniform magnetisch veld opgewekt door een Helmholtz-spoel.  
Door het vari\"eren van de stroom die door de spoel of de spanning die de 
elektronen versnelt proberen we zo de meting uit te voeren.

\section{Voorbereiding}

\subsection{Helmholtz spoel}

Een Helmholtz spoel is een apparaat dat gebruikt wordt om zeer uniforme 
magnetische velden aan te leggen. Het is opgebouwd uit twee spoelen met straal 
$R$ en $N$ windingen. Ze bevinden zich op een afstand $R$ van elkaar,
hetzelfde als de straal. Door beide spoelen stroomt dezelfde stroom $I$. De 
opstelling staat afgebeeld in figuur \ref{helmholtz}.

%\figuur{width=0.4\textwidth}{helmholtz}{Schematische weergave van een 
%Helmholtz spoel}

Het veld van de opstelling is de som van de velden van twee aparte spoelen. Het 
veld van \'e\'en enkele spoel op de centrale as kan berekend worden met de wet 
van Biot-Savart.
$$
\vec{B} = \frac{\mu_0 I}{4 \pi} \int \frac{\vec{dl} \times \hat{r}}{r^2}
$$
Vanuit symmetrieoverwegingen kan het $B$-veld enkel volgens de $x$-as zijn. Het 
veld op een afstand $h$ van het centrum wordt dus
$$
B_x = \frac{\mu_0 I}{4 \pi} \frac{2 \pi R}{r} \frac{R}{r}
$$
Hier wordt de uitdrukking voor $r$, de afstand tot een elementje van de spoel, 
ingevuld en de vergelijking wordt vereenvoudigd
$$
B = \frac{\mu_0 I R^2}{2(R^2+x^2)^{3/2}}
$$
Het veld in de Helmholtz spoel kan nu berekend door het veld van twee spoelen 
op te tellen. Omdat we het veld in het midden van het apparaat nodig hebben, 
stellen we $x$ gelijk aan $R/2$
$$
B = 2 \frac{\mu_0 I R^2}{{2(R^2+(R/2)^2})^{3/2}}
$$
Na vereenvoudiging bekomt men
$$
B = \left(\frac{4}{5}\right)^{3/2} 
$$

\subsection{Baan van een elektron in een uniform magnetisch veld}

De beginsnelheid van het elektron kan bepaald worden door de wet van behoud van 
energie.  De elektronen worden vanuit rust versneld door het aangelegde 
spanningsverschil.  De kinetische energie kan dus beschreven worden door
$$
\frac{mv^2}{2} = eV
$$
Oplossen naar de snelheid geeft
$$
v = \sqrt{ \frac{e}{m} 2V }
$$
De kracht op een bewegende lading $q$ in een elektromagnetisch veld wordt 
gegeven door de Lorentz-kracht
$$
\vec{F} = q(\vec{E} + \vec{v} \times \vec{B})
$$
In deze proef is enkel een magnetisch veld aanwezig. De lading van het elektron 
wordt voorgesteld door $e$. Tenslotte worden de elektronen loodrecht op de 
magnetische veldlijnen ge\"injecteerd. De uitdrukking voor de kracht wordt dan
$$
F = e v B
$$
Deze kracht staat altijd loodrecht op zowel de snelheid van het deeltje als het 
magnetisch veld. Omdat de kracht altijd loodrecht op de snelheid staat, blijft 
de grootte van de snelheid constant en de grootte van de kracht dus ook. De 
elektronen voeren een cirkelbeweging uit. In een cirkelbaan is de centrifugale 
kracht even groot als de centripetale kracht, in dit geval de magnetische 
kracht.
$$
\frac{mv^2}{r} = evB
$$
Waarbij $r$ de straal is van de cirkelbaan.
Daar de snelheid van de elektronen altijd onder 10\,\% van de lichtsnelheid 
blijft, mogen relativistische effecten verwaarloosd worden. Deze vergelijking 
wordt opgelost naar de soortelijke lading $e/m$
$$
\frac{e}{m} = \frac{v}{Br}
$$
Hier kan de uitdrukking voor snelheid rechtsreeks ingevuld worden omdat de 
snelheid constant blijft.
$$
\frac{e}{m} = \sqrt{\frac{e}{m} \frac{2V}{B^2r^2}}
$$
Hieruit volgt de uitdrukking voor de soortelijke lading.
$$
\frac{e}{m} = \frac{2V}{B^2r^2}
$$



\section{Metingen}
\begin{table}
\caption{Waargenomen straal van de elektronenbaan}
\label{tabel-elektron}
\begin{center}
\begin{tabular}{c|c|r@{.}l|r@{.}l|r@{.}l|r@{.}l}
\multicolumn{1}{c|}{$V$ (V)}&
\multicolumn{1}{c|}{$\Delta V$ (V)}&
\multicolumn{2}{c|}{$I$ (A)}&
\multicolumn{2}{c|}{$\Delta I$ (A)}&
\multicolumn{2}{c|}{$r$ (mm)}&
\multicolumn{2}{c}{$\Delta r$ (mm)}\\\hline
175&	3&	3&00&	0&05&	17&8&	0&2\\
175&	3&	2&50&	0&04&	22&5&	0&2\\
175&	3&	2&45&	0&04&	22&8&	0&2\\
175&	3&	2&00&	0&03&	27&5&	0&2\\
175&	3&	1&85&	0&03&	30&8&	0&2\\
175&	3&	1&50&	0&03&	35&8&	0&2\\
175&	3&	1&00&	0&03&	56&9&	0&2\\\hline
265&	4&	1&75&	0&03&	38&8&	0&2\\
225&	3&	1&75&	0&03&	36&0&	0&2\\
200&	3&	1&75&	0&03&	33&3&	0&2\\
175&	3&	1&75&	0&03&	31&3&	0&2\\
150&	3&	1&75&	0&03&	30&0&	0&2\\
125&	3&	1&75&	0&03&	27&0&	0&2\\
100&	3&	1&75&	0&03&	24&0&	0&2\\\hline
250&	4&	1&10&	0&03&	61&3&	0&2\\
250&	4&	1&30&	0&03&	52&5&	0&2\\
250&	4&	1&50&	0&03&	44&3&	0&2\\
250&	4&	1&70&	0&03&	37&8&	0&2\\
250&	4&	1&90&	0&03&	35&0&	0&2\\
250&	4&	2&10&	0&03&	32&3&	0&2\\
250&	4&	2&30&	0&03&	28&8&	0&2\\
250&	4&	2&50&	0&04&	24&8&	0&2\\
250&	4&	2&75&	0&04&	18&3&	0&2\\
250&	4&	3&00&	0&05&	19&3&	0&6\\\hline
245&	4&	1&10&	0&03&	60&5&	0&2\\
225&	3&	1&10&	0&03&	58&3&	0&2\\
200&	3&	1&10&	0&03&	55&3&	0&2\\
175&	3&	1&10&	0&03&	51&0&	0&2\\
150&	3&	1&10&	0&03&	47&3&	0&2\\
125&	3&	1&10&	0&03&	42&0&	0&2\\
100&	3&	1&10&	0&03&	37&5&	0&2\\
\end{tabular}
\end{center}
\end{table}


Er werden vier reeksen van metingen verricht. De eerste twee metingen 
werden uitgevoerd bij constante spanning. De stroom werd gevari\"eerd 
waarbij de straal telkens gemeten werd. Bij de laatste twee reeksen werd 
daarentegen de stroom constant gehouden. De resultaten van de metingen zijn 
ondergebracht in tabel \ref{tabel-elektron}.

Volgens het afgeleide model (\ref{em-equation}) is de straal recht 
evenredig met de vierkantswortel van het aangelegde spanningsverschil en 
omgekeerd evenredig met de stroom. Via een lineaire regressie met 
respectievelijk de vierkantswortel van de aangelegde spanning en de inverse 
van de stroom is dit verband te onderzoeken. Theoretisch gezien zou de 
intercept met de $x$-as gelijk moeten zijn aan nul.  De bekomen waarden 
voor de metingen bij constante spanning zijn
$$
a_1 = (0.02 \pm 0.02)\,(\textrm{Am})^{-1}
\qquad \textrm{en} \qquad
a_2 = (0.08 \pm 0.02)\,(\textrm{Am})^{-1}
$$
met $a_1$ de offset voor de meting bij een spanning van $XX$\,V en $a_2$ de 
offset voor de meting bij $YY$\,V. Bij de meting met constante spanning 
horen de waarden
$$
a_3 = (-0.4 \pm 0.4)\,\textrm{V}^{1/2}\textrm{/m}
\qquad \textrm{en} \qquad
a_4 = (1.0 \pm 0.3)\,\textrm{V}^{1/2}\textrm{/m}
$$
waarbij $a_3$ de offset is voor de meting bij een constante stroom van 
$XX$\,A en $a_4$ deze bij een stroom van $YY$\,A.




Merk op dat een lineaire regressie enkel rekening houdt met de statistische 
fout. Er komt echter nog een behoorlijke fout bij ten gevolge van een fout 
op de metingen, zoals ook later zal blijken voor de gevonden 
$\mathcal{Q}$-waarde.

De intercepts blijken hier, met inachtneming van de instrumentele fouten, 
klein genoeg dat we mogen aannemen dat ze nul zijn.

Daarom


\end{document}
