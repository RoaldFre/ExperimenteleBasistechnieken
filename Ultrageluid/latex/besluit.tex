\section{Besluit}
Zowel met de puls-echo methode als met de resonantiemethode wordt een 
waarde voor de geluidssnelheid in water gevonden die lichtjes afwijkt van 
de literatuurwaarde.  Beide methodes zijn echter wel in overeenstemming met 
elkaar over de gevonden waarde.  Hierbij is de resonantimethode iets 
nauwkeuriger (zoals verwacht).  Bovendien is de fout op de 
resonantiemethode dan nog conservatief geschat.

De afwijking ten opzichte van de literatuurwaarde kan eventueel zijn 
binnengeslopen door een contaminatie van het (gedestilleerde) water met 
eerder gemeten vloeistoffen, een licht afwijkende lengte van de 
resonantiekamer of een afwijkende waarde van de repetitiefrequentie van de 
burst. Eventueel kunnen ook diepere verschijnselen zoals een (gradi\"ent 
in) de hydrostatische druk, of eventuele opgeloste gassen in het water een 
rol hebben gespeeld.

Een andere factor die de metingen be\"invloedde is de sterke gevoeligheid 
voor de kleinste opwarming. Dit impliceert ook dat de metingen snel moesten 
gebeuren, wat niet evident was met de gebruikte manuele meetmethode.

Een van de nadelen van zulke precieze methoden is dus dat er veel aandacht 
moet besteed worden aan uitwendige factoren. Zij moeten immers elk even 
precies bepaald worden (en stabiel blijven) opdat het uiteindelijke 
resultaat van de nauwkerige meetmethode nog zin zou hebben.
