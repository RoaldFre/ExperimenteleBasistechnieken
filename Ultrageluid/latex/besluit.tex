\section{Besluit}
Zowel met de puls-echo methode als met de resonantiemethode wordt een 
waarde gevonden die lichtjes afwijkt van de literatuurwaarde. Beide 
methodes zijn bovendien in overeenstemming met elkaar over de gevonden 
waarde.

De afwijking kan eventueel zijn binnengeslopen door een contaminatie van 
het (gedestilleerde) water met eerder gemeten vloeistoffen, een licht 
afwijkende lengte van de resonantiekamer of een afwijkende waarde van de 
repetitiefrequentie van de burst. Eventueel kunnen ook diepere 
verschijnselen zoals een (gradient in) de hydrostatische druk, of eventuele 
opgeloste gassen in het water een rol hebben gespeeld.

In elk geval leiden beide methoden tot eenzelfde resultaat. Hierbij is de 
resonantimethode iets nauwkeuriger (zoals verwacht). Bovendien is de fout 
op de resonantiemethode dan nog conservatief geschat.

Een van de nadelen van zulke precieze methoden is dus dat er veel aandacht 
moet besteed worden aan uitwendige factoren. Zij moeten immers elk even 
precies bepaald worden opdat het uiteindelijke resultaat van de nauwkerige 
meetmethode nog zin zou hebben.


warmte snel want nauwk
