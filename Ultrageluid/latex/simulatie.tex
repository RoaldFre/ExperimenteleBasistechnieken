\section{Numerieke simulatie}
Als extra voorbereiding werd een numerieke simulatie uitgevoerd. Alle 
bovenstaande plots zijn hier resultaten van.

De simulatie werd opgezet met waarden voor de parameters die overeenkomen 
met de experimentele waarden in de volgende sectie.

Voor de eenvoud werd wel gekozen om de sinusburst te moduleren door een 
herhaalde gaugssische, en niet door een blokgolf. Dit laat toe om 
dispersieverschijnselen eenvoudiger te simuleren. Concreet werd de 
dispersie ge\"implementeerd door de standaarddeviatie van de omhullende 
gaussische met een constante snelheid in de tijd te laten verbreden.

Per reflectie wordt de breedte van de gaussische zo met een constante 
waarde verhoogd. Bovendien wordt de hoogte van een puls per `trip` (dus per 
twee reflecties) verlaagd door te vermenigvuldigen met een 
reflectieco\"effici\"ent. In de simulatie werd hiervoor een waarde van 
$0.7$ gekozen, dit komt ruwweg overeen met de experimenteel geobserveerde 
waarde.

Een compleet golfpakket met interferentie van haar reflecties wordt dan 
opgbouwd door een burst op te tellen met haar reflecties (met grotere 
breedte en lagere hoogte) die werden verschoven over een kleine 
delayperiode afhankelijk van het verschil tussen de repetitiefrequentie van 
de bursts en de resonantiefrequentie. De opbouw van zo'n pakket word 
weergegeven in figuur \ref{comboPakket}. Onderaan staat het `eindpakket' 
dat bekomen is door optellen van de bovenstaande gereflecteerde bursts (in 
het vervolg van de simulatie zal steeds gewerkt worden met een som van 20 
reflecties). De afname in amplituden ten gevolge van de 
reflectieco\"effici\"ent en de verbreding van de puls ten gevolge van 
dispersie zijn duidelijk merkbaar per reflectie.

\figuurOctave{comboPakket}{Samenstellen van een golfpakket door 
interferentie tussen de uitgezonde puls en haar reflecties}

In wat volgt zal met `delay' steeds het tijdsverschil bedoeld tussen het 
moment waarop een \'e\'en maal gereflecteerde puls aankomt bij een nieuw 
uitgezonde puls ten opzichte van het tijdsverschil voor ditzelfde traject 
voor dezelfde pulsen indien de burstrepetitie zo gekozen is dat deze twee 
pulsen perfect constructief interfereren (bovenstaand besproken figuur zou 
dit duidelijk moeten maken).

Door nu deze delay te laten vari\"eren rond nul wordt gesimuleerd wat er 
gebeurt indien de burstrepetitiefrequentie vari\"eert rond de 
resonantiefrequentie. In figuur \ref{delayinterferenties} wordt getoond hoe 
het uiteindelijke golfpakket zich dan gedraagt. Wat we zien in het midden 
(met de $y$-index 0) is de resonantiefrequentie (de centrale piek op figuur 
\ref{delayamp}), de twee grote pakketten op $y$-index $-4$ en $4$ zijn de 
twee pieken naast de centrale piek van figuur \ref{delayamp}. De paketten 
ertussen bevinden zich in het gebied met overheersend destructieve 
interferentie.

Uit de maxima van de amplituden van de golfpaketten uit figuur 
\ref{delayinterferenties} is dan ook figuur \ref{delayamp} kunnen 
gesimuleerd worden. Deze maximale piekamplituden werden in deze laatste 
figuur geplot in functie van de delaytijd voor 1 `trip' van een burst.



\figuurOctave{delayinterferenties}{Gedrag van het uiteindelijke golfpakket 
bij variatie van de delay of burstrepetitiefrequentie}




