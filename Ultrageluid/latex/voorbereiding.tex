\subsection{Model}

In vervormbare materialen is er voor kleine vervormingen een lineair verband 
tussen de relatieve volumeverandering en de druk nodig voor deze 
volumeverandering. De verhoudingsconstante is de bulkmodulus en is eigen aan 
elk materiaal.
\begin{equation}
\label{bulkmod}
B = -\frac{\Delta P}{\frac{\Delta V}{V}} = - V \frac{\partial P}{\partial V}
\end{equation}

\figuur{width=0.4\textwidth}{bulk}{Bulk modulus}


\subsection{Puls-echo methode}
De meest voor de hand liggende manier om de geluidssnelheid te bepalen is meten 
hoe lang het geluid erover doet om een bepaalde afstand te overbruggen. Als de 
tijd van het uitzenden tot het ontvangen van puls $\Delta t$ is en de afstand 
ertussen $L$ is, is de snelheid eenvoudigweg
$$
v = \frac{L}{\Delta t}
$$
In de resonantiekamer kan een enkele puls enkele keren heen en weer blijven 
reflecteren. Telkens reflecteert een vaste fractie van de amplitude, de 
reflectieco"effici"ent. Ook worden de golfpakketjes verstrooid door dispersieve 
effecten. Een typische waarneming werd gesimuleerd en weergegeven in figuur 
\ref{pulsecho}.  Een voordeel van het gebruiken van een resonantiekamer is dat
men meerdere waarnemingen in \'e\'en keer kan doen.

%%%% ROALD? %%%%%%%%%%%
%% Figuur van de uitstervende echos
%%%%%%%%%%%%%%%%%%%%

Deze methode heeft slechts een beperkte nauwkeurigheid.  De geluidsbron en de 
ontvanger moeten zeer goed gesynchroniseerd worden. Door de afstand te 
vergroten heeft het geluid meer tijd nodig, maar het wordt dan moeilijker om 
alle omstandigheden te controleren. 

\subsection{Staande golf methode}

In een resonantiekamer kan men een verfijndere methode gebruiken. Er worden 
geluidsgolven uitgestuurd van een zeer hoge frequentie, in de orde van enkele 
megahertz. Deze draaggolf wordt gemoduleerd door een blokgolf tot pulsen met 
een frequentie die rond de resonantiefrequentie zit. De uitgezonden puls 
reflecteert heen en weer in de kamer en verliest zijn energie maar dit keer 
worden er ook nieuwe pulsen uitgestuurd. Afhankelijk van wanneer de pulsen 
uitgestuurd worden, interfereren ze constructief of destructief met elkaar. De 
sterkste constructieve interferentie doet zich voor wanneer de frequentie van 
de uitgezonden pulsen \'e\'en van de resonantiefrequenties zijn. Deze 
interferentie doet zich voor wanneer een puls een geheel aantal keer de lengte 
van de resonantiekamer heeft afgelegd en is teruggekeerd voor een nieuwe puls 
wordt uitgestuurd.
\begin{equation}
\label{resfreq}
f_n = \frac{1}{T_n} = \frac{1}{n} \frac{v}{2L}
\end{equation}
In het ideale geval zou elke puls uit slechts \'e\'en cyclus van de draaggolf 
hoeven bestaan. Men zou dan die puls enkel moeten laten overeenkomen met zijn 
reflecties om resonantie waar te nemen. Vanwege praktische redenen is dit 
echter niet mogelijk.

Een puls wordt gedempt door een reflectie. Om de interferentie te kunnen 
waarnemen, zijn we dus aangewezen een grotere hoeveelheid energie uit te zenden 
met elke puls. Ook blijft het geen mooi geheel. Door dispersie wordt de puls 
verstrooid wat de waarnemingen bemoeilijkt. Bredere pulsen hebben hier minder 
last van.

%% ROALD %%
% Figuur van disperserende puls?
%%%%%%%%%%%

Anderzijds mag men ook niet teveel cyclussen per puls uitzenden. Het wordt dan 
moeilijker om de resonantie-effecten van het uitgezonden signaal te 
onderscheiden. In de praktijk gebruikt men een arbeidscyclus ( dutycycle ) van 
10 \`a 20 \%.  


%%%% KASPER
%%% AFWERKEN
%%% Figuur van amplitudeverloop


