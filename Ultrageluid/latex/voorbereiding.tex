\section{Voorbereiding}
\subsection{Model}

In vervormbare materialen is er voor kleine vervormingen een lineair verband 
tussen de relatieve volumeverandering en de druk nodig voor deze 
volumeverandering. Voor een illustratie, zie figuur (\ref{bulk}). De 
verhoudingsconstante is de bulkmodulus en is eigen aan elk materiaal.
\begin{equation}
\label{bulkmod}
B = -\frac{\Delta P}{\frac{\Delta V}{V}} = - V \frac{\Delta P}{\Delta V}
\end{equation}
\figuur{width=0.7\textwidth}{bulk}{Bulk modulus}

In dit model wordt er een cylinderkolom van een vloeistof beschouwd. De 
uitwijkingen zijn enkel volgens de $x$-co\"ordinaat. De uitwijking van de 
vloeistofdeeltjes oorspronkelijk op co\"ordinaat $x$ en op het tijdstip $t$ 
wordt voorgesteld door $u(x,t)$. Een infinitesimale cylinder van lengte $dx$ en 
oppervlakte $A$ heeft als volume $A dx$.  De uitwijking van de druk $P$ boven 
de atmosferische druk $P_0$ is $p$
$$
p = \Delta P = P - P_0
$$

Hiermee kan de uitdrukking voor de bulkmodulus (\ref{bulkmod}) voor dit model 
herschreven worden tot
$$
B = -A dx \frac{\Delta P}{\Delta V} = -dx \frac{\Delta P}{ u(x+dx,t)-u(x,t)}
$$
De uitdrukking voor de drukuitwijking kan hieruit ge\"isoleerd worden
\begin{equation}
\label{bulkmod1d}
p = \Delta P = -B \frac{u(x+dx,t)-u(x)}{dx} = -B \partieel{u}{x}
\end{equation}

De wet van Newton ($F = ma$) voor de cylinder wordt
$$
P(x)A - P(x+dx)A = (p(x) - p(x+dx))A = \rho A dx \partpart{u}{t}
$$
Hierbij is $\rho$ de dichtheid van de vloeistof
$$
\partieel{p}{x} + \rho \partpart{u}{t} = 0
$$
Hierin kan uitdrukking (\ref{bulkmod1d}) gesubstitueerd worden
$$
-B\partpart{u}{x} + \rho \partpart{u}{t} = 0
$$
Na herschrijven wordt dit
\begin{equation}
\label{waveeq}
\partpart{u}{x} - \frac{\rho}{B} \partpart{u}{t} = 0
\end{equation}

Dit is de in de natuurkunde alomtegenwoordige golfvergelijking. De snelheid van 
een golf is eenvoudigweg
\begin{equation}
\label{speedofsound}
v = \sqrt{\frac{B}{\rho}}
\end{equation}

De waardes voor de dichtheid en de bulkmodulus kunnen worden opgezocht. Zo kan 
men voor een materiaal de geluidssnelheid berekenen. De resultaten voor enkele 
materialen staan in tabel \ref{speedtable}. 

\begin{table}[htb]
\caption{Berekende geluidsnelheid}
\label{speedtable}
\begin{center}
\begin{tabular}{c||lcc}
Materiaal & $B$ (Pa) & $\rho$ $(\textrm{kg/m}^3)$ & $v$ (m/s) \\\hline
lucht	& $1.42 \cdot 10^5$	& 1.2  & 344 \\
water	& $2.18 \cdot 10^9$	& 1000 & 1477 \\
glycerol& $4.35 \cdot 10^9$	& 1200 & 1900 \\
staal	& $1.60 \cdot 10^{11}$	& 7900 & 4500 \\
\end{tabular}
\end{center}
\end{table}

\subsection{Puls-echo methode}
De meest voor de hand liggende manier om de geluidssnelheid te bepalen is meten 
hoe lang het geluid erover doet om een bepaalde afstand te overbruggen. Als de 
tijd van het uitzenden tot het ontvangen van puls $\Delta t$ is en de afstand 
ertussen $L$ is, is de snelheid eenvoudigweg
$$
v = \frac{L}{\Delta t}
$$
In de resonantiekamer kan een enkele puls een aantal keer heen en weer 
reflecteren. Telkens heeft de gereflecteerde puls maar een fractie van de 
amplitude van de oorspronkelijke puls, de verhouding gegeven door de 
reflectieco"effici"ent.  Ook worden de golfpakketjes verstrooid door 
dispersieve effecten. Een typische waarneming werd gesimuleerd en weergegeven 
in figuur \ref{puls-echo}. 

\figuurOctave{puls-echo}{Gereflecteerde echo's met een reflectieco"effici\"ent 
van 0.7}

Een voordeel van het gebruiken van een resonantiekamer is dat
men meerdere waarnemingen in \'e\'en keer kan doen.
Deze methode heeft slechts een beperkte nauwkeurigheid. Door dispersie 
verstrooien de pulsen waardoor ze moeilijker te localiseren zijn.

\subsection{Resonantie methode}

In een resonantiekamer kan men een verfijndere methode gebruiken. Er worden 
geluidsgolven uitgestuurd van een zeer hoge frequentie, in de orde van enkele 
megahertz. Deze draaggolf wordt gemoduleerd door een blokgolf tot pulsen met 
een frequentie die rond de resonantiefrequentie zit. De uitgezonden puls 
reflecteert heen en weer in de kamer en verliest zijn energie maar dit keer 
worden er ook nieuwe pulsen uitgestuurd. Afhankelijk van wanneer de pulsen 
uitgestuurd worden, interfereren ze constructief of destructief met elkaar. De 
sterkste constructieve interferentie doet zich voor wanneer de frequentie van 
de uitgezonden pulsen \'e\'en van de resonantiefrequenties zijn. Dit valt voor 
wanneer een puls een geheel aantal keer de lengte van de resonantiekamer heeft 
afgelegd en is teruggekeerd voor een nieuwe puls wordt uitgestuurd.
\begin{equation}
\label{resfreq}
f_n = \frac{1}{T_n} = \frac{1}{n} \frac{v}{2L}
\end{equation}
De fundamentele resonantiefrequentie noemen we $\freso$.
$$
\freso = f_1 = \frac{v}{2L}
$$
Als men nu de resonantiefrequentie experimenteel kan bepalen kan men hieruit de 
snelheid berekenen
\begin{equation}
\label{v-ifv-f}
v = 2nLf_n = 2L \freso
\end{equation}
In het ideale geval zou elke puls uit slechts \'e\'en cyclus van de draaggolf 
hoeven bestaan. Men zou dan die puls enkel moeten laten overeenkomen met zijn 
reflecties om resonantie waar te nemen. Vanwege praktische redenen is dit 
echter niet mogelijk.

Een puls wordt gedempt door een reflectie. Om de interferentie te kunnen 
waarnemen, zijn we dus aangewezen een grotere hoeveelheid energie uit te zenden 
met elke puls. Ook blijft het geen mooi geheel. Door dispersie wordt de puls 
verstrooid wat de waarnemingen bemoeilijkt. Een simulatie van een disperserende 
puls wordt weergegeven in figuur \ref{dispersie}. Bredere pulsen hebben hier 
minder last van.

\figuurOctave{dispersie}{Impressie van een disperserende puls}

Anderzijds mag men ook niet teveel cycli per puls uitzenden. Het wordt dan 
moeilijker om de resonantie-effecten van het uitgezonden signaal te 
onderscheiden. De resonantiepiek wordt ook platter. In de praktijk gebruikt men 
een arbeidscyclus (dutycycle) van 10 \`a 20\,\%.


%%%% KASPER
%%% AFWERKEN
%%% Figuur van amplitudeverloop
%%%%%%


