\section{Proefopstelling}

\figuur{width=1.0\textwidth}{opstelling}{blabla}

De proefopstelling wordt weergegeven in figuur \ref{opstelling}. Het centrale 
deel van de opstelling is de resonantiekamer. Dit is een cylindervormige ruimte 
van precies 5\,cm lang. Aan \'e\'en kant van de cylinder wordt een 
geluidssignaal opgewekt dat aan de andere kant kan opgevangen worden. De 
cylindervlakken reflecteren een groot deel van de puls terug naar de andere 
kant. 

Een computer regelt de uitgangsfrequentie van een sinusgenerator. In dit 
experiment word er gewerkt met frequenties van 13 tot 16\,kHz. Dit signaal 
dient als trigger voor de functiegenerator. De functiegenerator werk op een 
frequentie van 2.1\,MHz. Het stuurt een burst uit van sinusgolven, 
overeenkomstig met modulatie door een blokgolf. Op de functiegenerator is de 
duty cycle in te stellen door het aantal uitgestuurde cycli in te geven per 
burst. In dit experiment wordt er gewerkt met een duty cycle van 15\,%.
