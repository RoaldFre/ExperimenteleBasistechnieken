\section{Proefopstelling}

\figuur{width=0.85\textwidth}{opstelling}{Proefopstelling}

De proefopstelling wordt weergegeven in figuur \ref{opstelling}. Het centrale 
deel van de opstelling is de resonantiekamer. Dit is een cylindervormige ruimte 
van precies 5\,cm lang. Aan \'e\'en kant van de cylinder wordt een 
geluidssignaal opgewekt dat aan de andere kant kan opgevangen worden. De 
cylindervlakken reflecteren een groot deel van de puls terug naar de andere 
kant. 

Een computer regelt de uitgangsfrequentie van een sinusgenerator. In dit 
experiment word er gewerkt met frequenties van 13 tot 16\,kHz. Dit signaal 
dient als trigger voor de functiegenerator. De functiegenerator werkt op een 
frequentie van 2.1\,MHz. Het stuurt een burst uit van sinusgolven, 
overeenkomstig met modulatie door een blokgolf. Op de functiegenerator is de 
duty cycle in te stellen door het aantal uitgestuurde cycli in te geven per 
burst.

Voor de puls-echo methode wordt er een enkele puls uitgestuurd aan 1 kHz. Zo 
wordt een continue uitlezing gegarandeerd maar kan een puls ook nooit met de 
volgende puls interfereren. Op de oscilloscoop worden dan waarnemingen gedaan 
naar de tijd tussen elke reflectie.

De andere methode is de resonantiemethode. Er worden pulsen uitgestuurd met een 
duty cycle van ongeveer 15\,\% aan frequenties rond de resonantiefrequentie 
$f_reso$. Door de frequentie van de sinusgenerator op de computer aan te 
passen, kan me zo waargenomen amplitude veranderen. Op deze manier worden alle 
locale maxima gevonden rond de resonantiefrequentie. Met de beschreven methode 
kan zo de geluidssnelheid berekend worden. 
