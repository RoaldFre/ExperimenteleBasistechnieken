\newcommand{\relPiekAmp}{
\begin{table}[htb]
\caption{Relatieve piekamplituden bij de resonantiefrequenties}
\label{relPiekAmp}
\begin{center}
\begin{tabular}{ r | r@{}l | r@{}l  ||  r | r@{}l | r@{}l }
\multicolumn{1}{c|}{$f$ (Hz) $\pm 4$}&
\multicolumn{2}{c|}{$A$}&
\multicolumn{2}{c||}{$\Delta A$}&
\multicolumn{1}{c|}{$f$ (Hz) $\pm 4$}&
\multicolumn{2}{c|}{$A$}&
\multicolumn{2}{c}{$\Delta A$}\\\hline
12870&	0&.57&	0&.01&	14498&	0&.91&	0&.03\\
12979&	0&.59&	0&.01&	14600&	0&.85&	0&.03\\
13056&	0&.59&	0&.01&	14702&	0&.81&	0&.03\\
13138&	0&.60&	0&.03&	14814&	0&.79&	0&.03\\
13222&	0&.62&	0&.03&	14914&	0&.78&	0&.03\\
13306&	0&.66&	0&.03&	15016&	0&.76&	0&.03\\
13388&	0&.72&	0&.03&	15121&	0&.72&	0&.03\\
13470&	0&.76&	0&.03&	15207&	0&.71&	0&.03\\
13552&	0&.79&	0&.03&	15320&	0&.69&	0&.03\\
13638&	0&.81&	0&.03&	15430&	0&.68&	0&.03\\
13728&	0&.82&	0&.03&	15542&	0&.66&	0&.03\\
13818&	0&.85&	0&.03&	15656&	0&.65&	0&.03\\
13912&	0&.88&	0&.03&	15772&	0&.62&	0&.03\\
14006&	0&.91&	0&.03&	15888&	0&.60&	0&.01\\
14102&	0&.97&	0&.03&	16010&	0&.59&	0&.01\\
14198&	1&.00&	0&.03&	16134&	0&.58&	0&.01\\
14296&	1&.00&	0&.03&	16256&	0&.57&	0&.01\\
14396&	0&.94&	0&.03&	16384&	0&.56&	0&.01\\
\end{tabular}
\end{center}
\end{table}
}


\newcommand{\waterDelay}{
\begin{table}[htb]
\caption{Echo delay meting voor water}
\label{waterDelay}
\begin{center}
\begin{tabular}{ r | r@{}l | r@{}l }
\multicolumn{1}{c|}{$n$}&
\multicolumn{2}{c|}{$v$ (km/s)}&
\multicolumn{2}{c}{$\Delta v$ (km/s)}\\\hline
6&	1&.40&	0&.03\\
9&	1&.41&	0&.02\\
11&	1&.43&	0&.02\\
13&	1&.43&	0&.02\\
\end{tabular}
\end{center}
\end{table}
}





\section{Metingen voor water}
[proefopstelling zegt al: 5cm lengte]

Zeggen: modulatie door BLOKgolf, geen gaussische!

\subsection{Verloop van de piekamplituden van het golfpakket nabij de 
resonantiefrequentie}
Eerst werd het gedrag van de piekamplituden van het golfpakket onderzocht 
indien de sinusoidale bursts worden afgevuurd aan een frequentie die in de 
buurt ligt van de resonantiefrequentie. Met `de piekamplitude van het 
golfpakket' wordt de maximale amplitude van een beschouwd golfpakket 
bedoeld, terwijl dit interferentie ondervindt van de reflecties van vorig 
uitgestuurde bursts. In figuur \ref{comboPakket} betekent dit de maximale 
amplitude voor het onderste, samengestelde pakket.

Het doel van deze meting is om te onderzoeken wat de vorm is van de 
omhullende van het patroon (dat een intrinsieke snelle schommeling zal 
kennen door het afwisselen van destructieve en constructieve interferentie) 
en om eventueel dit patroon te parametriseren en zo met behulp van een fit 
een zeer nauwkeurige waarde te bekomen voor de resonantiefrequentie, en dus 
de geluidssnelheid.

Praktisch werd de triggering eerst zo ingesteld dat er aan een frequentie 
van $13.5$\,kHz een burst van van 25 sinusperioden (aan een frequentie van 
$2.1$\,MHz) werd afgevuurd. Op de oscilloscoop was te merken dat deze 
triggeringsfrequentie iets onder de resonantiefrequentie lag. Het 
pulspakket (na interferentie met de gereflecteerde bursts) had immers een 
vorm zoals het onderste pakket in figuur \ref{delayinterferenties}. De 
`linkerstaart' was wel nog wat `zwaarder' dan in die figuur (de frequentie 
lag verder weg van het optimum).

Vervolgens werd de triggeringsfrequentie verhoogd in stappen van 5\,Hz om 
een gedetailleerd beeld te krijgen van het verloop van deze piekamplitude 
bij afwisselend constructieve en destructieve interferentie met de 
weerkaatste golfpaketten.

\figuurOctave[htb]{delayamp-experimenteel}{Visualisatie van de 
piekamplituden bij veranderende burstfrequentie}
Het resultaat hiervan is geplot in figuur \ref{delayamp-experimenteel}.  
Merk op dat dit maal gekozen is om de meetpunten expliciet te verbinden met 
een lijnstuk om het verloop visueel duidelijk te maken.

Uit deze plot volgt inderdaad dat het gekozen frequentiegebied onder de 
resonantiefrequentie lag, de pieken in de gemeten piekamplitude blijven 
immers nog stijgen voor stijgende burstfrequenties.

Omwille van een aantal redenen is gekozen om dit type meting niet verder te 
zetten tot voorbij de resonantiefrequentie. Een eerste reden is puur 
praktisch: deze meting bevat zo'n 130 meetpunten en het verkrijgen van deze 
data via de gebrukte, manuele methode heeft een behoorlijke hoeveelheid 
tijd in beslag genomen. Er was eenvoudigweg onvoldoende tijd om deze 
resolutie in de bekeken burstfrequentie aan te houden tot voorbij de 
resonantiepiek.


Ten tweede zorgt de gebruikte gevoeligheid van de apparatuur ervoor dat 
kleine factoren de bekomen meetwaarden danig kunnen verstoren. Een mooi 
voorbeeld hiervan vinden we terug op de plot in het gebied rond 13650\,Hz.  
Er is duidelijk een uitschietende piek merkbaar, en het vervolg van de puls 
lijkt een beetje verschoven. Dit is gerelateerd aan de trage meetproces.  
Hoewel het water en de apparatuur bij het begin van de meting op 
kamertemperatuur was, heeft er tijdens de metingen blijkbaar een kleine 
opwarming plaatsgevonden die constant de plaats van de 
resonantiefrequenties deed verschuiven.

Deze opwarming kan eventueel puur afkomstig zijn van de aangeboden 
kinetische energie door de sinusbursts. De meetmethode is immers zo 
gevoelig dat even krachtig schudden met de cilinder voldoende was om de 
2\,MHz golf een volledige periode te laten verschuiven.

Dit impliceert dat de meetgegevens een onvermijdelijke bias zullen vertonen 
indien de duur van de meting te lang is. Bovendien moet aan eenzelfde tempo 
worden verdergewerkt om deze bias overal gelijk te houden.


Ten laatste zou deze data aan hoge frequentieresolutie geen meerwaarde 
hebben opgeleverd in vergelijking met de hieronder beschreven methode die 
enkel de pieken in deze plot zal meten.  Uit de plot blijk immers (net 
zoals uit de simulatie voor een gaussisch gemoduleerde burst) dat het 
onmogelijk is om de `inwendige' oscillaties te fitten met een eenvoudige 
sinusfunctie.  De toppen lijken immers ook hier veel sterker te worden 
versterkt dan de dalen (te merken aan hun hogere amplitude en kleinere 
breedte in de pieken). Een model hiervoor opstellen zou ons te ver leiden.



\subsection{Verloop van de maxima van de piekamplituden nabij de 
resonantiefrequentie}
Uit de opmerkingen bij de voorgaande methode werd besloten om enkel de 
maxima van het profiel van figuur \ref{delayamp-experimenteel} te bepalen.  
Dit resulteert in minder meetpunten, en dus een snellere meting. Hierdoor 
wordt de kans tot opwarming verlaagd. Bovendien is een kleine opwarming nu 
minder dramatisch, aangezien de plaats van het locale maximum van de 
piekamplitude van het golfpakket wel lichtjes zal wijzigen, maar het 
globale `maximum van deze maxima' zal nog steeds op ongeveer dezelfde 
plaats bijven.

Als extra voorzorg om het effect van opwarming te verminderen werd de 
apparatuur in een waterbad van $18.5\grad$ geplaatst.

De resultaten van deze meting zijn getabuleerd in tabel \ref{relPiekAmp} en 
grafisch weergegeven in figuur \ref{amp-experimenteel}. Omdat de werkelijke 
`ruwe' meetwaarde geen extra informatie levert (de specificaties van de 
sensor zijn niet gekend), is de gemeten amplitude relatief weergegeven ten 
opzichte van de maximaal gemeten waarde.

\relPiekAmp
\figuurOctave{amp-experimenteel}{Relatieve piekamplituden bij de 
resonantiefrequenties}

Een vergelijking met de maxima van de simulatie voor een gaussische 
sinusburst in figuur \ref{delayamp} is deze figuur veel minder symmetrisch.  
Ook de pulspaketten bij de resonantie hadden een asymmetrsiche vorm. Dit is 
eventueel te verklaren door kleine imperfecties in de resonantiekamer, en 
door imperfect botsingen aan de wanden (daar bevinden zich immers de zender 
en ontvanger).

Omdat de plot sterk afwijkt van een gaussische, zullen we niet trachten om 
hier een fit op uit te voeren. Slechts enkele punten nabij de piek in 
beschouwing nemen zou eventueel wel een goede fit kunnen opleveren, maar de 
kleine hoeveelheid meetpunten zou een vertekend beeld geven van de 
accuraatheid. Bovendien zou dan de relatief grote fout in de relatieve 
piekamplitude niet worden in acht genomen.

We besluiten hierom om gewoon vanuit de plot de meest waarschijnlijke 
waarde te bepalen. Het lijkt aannemelijk dat de echte piek zich tussen
14198\,Hz en 14296\,Hz bevind, waarbij de waarde dichter bij deze eerste 
zal liggen. Gezien de grootte van de errorbars, zullen we de waarde 
voorzichtig afschatten tot
$$
\freso = (14.2 \pm 0.1)\,\rm kHz
$$
Uit \ref{v-ifv-f} volgt dan voor de geluidssnelheid
$$
v = 2L\freso
= (1.42 \pm 0.01)\,\rm km/s
$$

De resonantiemethode levert dus een waarde voor de geluidssnelheid met een 
behoorlijke nauwkeurigheid van drie beduidende cijfers (met een kleine fout 
op het laatste cijfer).

De literatuurwaarde voor de snelheid van geluid in water bij $20\grad$C is 
$1.48\,$km/s. Onze gevonden waarde zit daar (veel) boven. Hoewel het water 
bij de proefneming een iets lagere temperatuur had ($18.5\grad$C), is deze 
afwijking te groot om verklaard de worden door enkel en alleen het 
temperatuursverschil. Een controle werd uitgevoerd in de vorm van de 
puls-echo methode.

\subsection{Vergelijking met de puls-echo methode}
Via de puls-echo methode werd een tweede maal de geluidssnelheid in water 
berekend. De resultaten worden weergegven in tabel \ref{delayWater}

\waterDelay

Uit de opgemeten gevens volgt een gemiddelde waarde van
$$
v = (1.42 \pm 0.02)\,\rm km/s
$$

Merk op dat de onzekerheid op deze waarde slechts iets groter is dan de 
fout op de gevonden geluidssnelheid via de resonatiemethode, hoewel deze 
laatste methode in theorie nauwkeuriger is (wel moet gezegd worden dat die 
fout eerder conservatief werd bepaald).

Bovendien is de gevonden waarde in goede overeenstemming met de waarde 
bekomen met de resonantiemethode. De gevonden afwijking is dus niet 
tengevolge van een meetfout.

Een mogelijke factor die deze afwijking kan verklaren is eventuele een 
kleine  factoren die mogelijk aan de basis van deze afwijking liggen

impurities
theorie: 1477 







